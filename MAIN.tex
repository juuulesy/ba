%***********************************************************************
%* File            :    MAIN.tex
%*
%* HAUPTDOKUMENT
%*
%* Dieses Dokument bindet alle untergeordneten Dokumente
%* ein. Es wird beim Aufruf von pdflatex als Quelle angegeben.
%*
%* Autoren         :    Daniel Hering, Tobias Wartzek
%*√
%* Datum           :    05.12.2009
%***********************************************************************



\documentclass[12pt,a4paper,twoside,openright,index=totoc,listof=totoc,DIV15,BCOR=8.25mm,headinclude,footinclude=false,headsepline,notitlepage,numbers=noenddot,bibliography=totoc,headings=normal,parskip=full]{scrbook}

%************************************************************************
% Das Package muss leider hier eingebunden werden da sonst in den
% Befehlen für hypersetup keinerlei Umlaute zur Verfügung stehen√
%************************************************************************

\usepackage{graphicx}
\usepackage{float}
\usepackage[utf8]{inputenc}         % erlaubt direkte Eingabe Deutscher
                                    % Sonderzeichen in .tex Dateien
                                    %(latin1=iso8859-1 fuer Unix Editoren
                                    % oder Windows Editoren die auch latin1
                                    % beherrschen)

%************************************************************************
% Dateiinfos eintragen (u.a. von hypersetup und f�r die Titelseite verwendet)
%************************************************************************
\usepackage{pdfpages}
% Titel der Arbeit
\newcommand*{\doclongtitle}{Bestimmung der Positionsbeziehungen von körpergetragenen Orientierungssensoren zur Beurteilung der posturalen Kontrolle}

% Art der Arbeit (Diplomarbeit, Studienarbeit, Bachelorarbeit, etc.)
\newcommand*{\docsubject}{Bachelorarbeit}

% Name des Autors
\newcommand*{\docauthor}{Julia Ribbrock}

% E-Mail Adresse des Autors
\newcommand*{\docauthoremail}{julia.ribbrock@rwth-aachen.de}

% Schlüsselworte der Arbeit
\newcommand*{\dockeywords}{MedIT Latex Studienarbeit Diplomarbeit Vorlage}

% Name des Betreuers
\newcommand*{\docsupervisor}{Markus Lüken, M.Sc.}

%\newcommand*{\docauthorfax}{+49 241 80-82442}
%\newcommand*{\docauthorphone}{+49 241 80-23211}

% Bild auf der Titelseite
\newcommand*{\doctitlepic}{images/title.jpg}

% folgende Befehle nicht ändern!!
\newcommand*{\dochomepage}{www.medit.hia.rwth-aachen.de}
\newcommand*{\docshorttitle}{\doclongtitle}

%**********************************************
%* Pakete (HEADER)                           *
%**********************************************

\include{HEADER}


%*************************************************************
%*************************************************************
%*																													**
%* 					Beginn des Dokuments 						  							**
%*																													**
%*************************************************************
%*************************************************************

\begin{document}
\begin{sloppypar}                       % schöner Blocksatz trotz langer Worte



%*************************************************************
%																														 *
%										 ENDE ALLER DEFINITIONEN 								 *
%																														 *
%*************************************************************


\frontmatter    												%römische Nummerierung aktivieren
\include{titelseite}
\cleardoubleemptypage


%**************
%* Danksagung *
%**************

\chapter{Danksagung}
Die Erstellung dieser Arbeit wäre mir ohne die Hilfe einiger Menschen nicht möglich gewesen.
Aus diesem Grunde möchte ich mich an dieser Stelle gern bei einigen Personen bedanken.

Ein großer Dank geht an meinen Betreuer, Herrn Markus Lüken (M.Sc.), der mir über die gesamte Bearbeitungszeit, jederzeit hilfreich zur Seite stand und mir, mit aufmunternden Worten, immer wieder zum nötigen Optimismus zur Fertigstellung dieser Arbeit verholfen hat.

Desweiteren möchte ich mich bei Herrn Prof. Dr.-Ing. Dr. Med. Leonhardt, dem Leiter des Lehrstuhls für medizinische Informationstechnik, für die Möglichkeit bedanken diese Arbeit am Institut verfassen zu können.

Meiner Familie möchte ich dafür danken, dass sie mich die letzten Jahre immer darin bekräftigt haben meinen Weg in diesem Studium zu gehen und nie aufzugeben.

\cleardoubleemptypage

%************************************************
%* Erklärung (hier muss nichts geändert werden) *
%************************************************

\include{erklaerung}
\cleardoubleemptypage

%************
%* Abstract *
%************

\include{abstract}
\cleardoubleemptypage

%**********************
%* Inhaltsverzeichnis *
%**********************


\addcontentsline{toc}{chapter}{Inhaltsverzeichnis} % Eintrag im Inhaltsverzeichnis erstellen
\tableofcontents                        % Inhaltsverzeichnis anlegen
\cleardoubleemptypage



%*********************
%* Symbolverzeichnis *
%********************

\chapter*{Symbolverzeichnis}										% da *-Variante m�ssen Kopfzeilen und TOC-Eintrag von Hand generiert werden
\markboth{Symbolverzeichnis}{Symbolverzeichnis} 				% Kopfzeile manuell anpassen
\addcontentsline{toc}{chapter}{Symbolverzeichnis}				% TOC-Eintrag



\section*{Abkürzungen}

%* Benutzung der Acronym-Umgebung
%* \acrodef{Name zum aufrufen im Dokument}[gew�nschte Abk�rzung]{ausgeschriebener Begriff}
%* \ac{Name} zur Verwendung im Dokument - beim 1.maligen Verwenden wird der ausgeschriebene
%* Begriff mit der Abk�rzung dahinter in Klammern gesetzt, danach nur noch die Abk�rzung

\acrodef{BSN}[BSN]{Body Sensor Network}
\acrodef{EKF}[EKF]{Extended Kalman Filter}
\acrodef{IMU}[IMU]{Inertial Measurement Unit}
\acrodef{MedIT}[MedIT]{Institut f{\"u}r medizinische Informationstechnik}
\acrodef{MEMS}[MEMS]{Microelectronic Mechanical Systems}
\acrodef{MN}[MN]{Masterknoten}
\acrodef{SN}[SN]{Sensorknoten}
\acrodef{ToF}[ToF]{Time of Flight}
\acrodef{UKF}[UKF]{Unscented Kalman Filter}
\acrodef{UT}[UT]{Unscented Transformation}
\acrodef{VBA}[VBA]{Vibrating Beam Accelerometer}


%* \acs{Name} ruft explizit die Abkürzung auf.
%* \acl{Name} ruft explizit den ausgeschriebenen Begriff auf.

\begin{tabularx}{\textwidth}{p{.18\textwidth}X}
\acs{BSN} & \acl{BSN} \\
\acs{EKF} & \acl{EKF} \\
\acs{IMU} &\acl{IMU} \\
\acs{MedIT} & \acl{MedIT} \\
\acs{MEMS} & \acl{MEMS} \\
\acs{MN} & \acl{MN} \\
\acs{SN} & \acl{SN} \\
\acs{ToF} & \acl{ToF} \\
\acs{UKF} & \acl{UKF} \\
\acs{UT} & \acl{UT} \\
\acs{VBA} & \acl{VBA} \\
\end{tabularx}

\cleardoubleemptypage

\mainmatter							% Arabische Nummerierung, Beginn des Hauptteils

%**************
%* Einleitung *
%**************

\chapter{Einleitung}
\section{Aufbau der Arbeit}
Die vorliegenden Arbeit befasst sich mit der Bestimmung der relativen Positionen der Sensorknoten innerhalb eines Body Sensor Networks.
Hierdurch soll eine Bewertung der posturalen Kontrolle des Trägers des Systems ermöglicht werden.

Das Kapitel 2 dient hierbei dazu, sowohl medizinische, mathematische, als auch technische Grundlagen zu erläutern.
Auf diie genauere Thematik und den verwendeten Ansatz dieser Arbeit wird in Kapitel 3 eingegangen.
Kapitel 4 befasst sich mit dem Messaufbau zur Validierung des Algorithmus mithilfe eines optischen Trackingsystems.
Die mithilfe der Messung erzielten Ergebnisse werden in Kapitel 5 näher diskutiert, um in Kapitel 6 schließlich zu einem Fazit über die verwendete Algorithmik und Sensorik zu gelangen.


%*************
%* Hauptteil *
%*************
% !TEX root = MAIN.tex

\chapter{Grundlagen}
\section{Medizinische Grundlagen}
\subsection{Posturale Kontrolle}
Der Begriff der \textit{posturalen Kontrolle} bezeichnet die Fähigkeit eines Menschen, auch trotz kurzfristiger Störungen, das motorische Gleichgewicht herstellen, aufrechterhalten und wiedererlangen zu können \cite{PK1}.
Da zur Realisierung der posturalen Kontrolle ein ständiger Abgleich zwischen dem sensorischen Prozess der Erfassung der Körperlage und eventueller Bewegungen und der motorischen Reaktion auf eben diese erfolgen muss, spricht man häufig auch von der sensomotirischen Gleichgewichtsregulation \cite{PK2}.

\section{Mathematische Methoden}
Innerhalb der vorliegenden Arbeit werden, unter Anderem zur Datenverarbeitung, verschiedene Berechnungen durchgeführt und mathematische Annahmen getroffen, welche auf grundlegenden mathematischen Ansätzen beruhen.
So auch vom Kalman Filter und einer seiner Weitereinwicklungen, dem Unscented Kalman Filter (UKF), Gebrauch gemacht.

\subsection{Kalman Filter}
Betrachtet man dynamische Systeme, muss stets davon ausgegangen werden, dass diese keine ideale Beschreibung des Systemverhaltens zulassen.
Mitunter werden die Systemgleichungen bei der mathematischen Betrachtung vereinfacht, was in einer, als Systemrauschen bezeichneten, additiven, nicht beeinflussbaren Störung des Systems resultiert.
Durch die Anbringung von Sensoren ist eine weitere Fehlerquelle gegeben, deren Einfluss über die Ausgangsgröße des Systems berücksichtigt wird. Störungen dieser Art werden Messrauschen genannt.
Sowohl das Systemrauschen, als auch das Messrauschen werden in Form von Zufallsprozessen in die Berechnungen miteinbezogen. Dies hat jedoch zur Folge, dass auch die Eingangs- und Ausgangsgrößen des Systems als stochastische Größen betrachtet werden müssen.
Um dennoch eine qualitativ hochwertige Aussage über die Zustände des Systems treffen zu können, muss ein, für den gestörten Eingangsvektor, optimaler Schätzwert gefunden werden. Als optimaler Wert wird hierbei, aufgrund der wahrscheinlichkeitstheoretischen Betrachtung, ein Wert minimaler Varianz angestrebt. \cite{Sys2}
\\Einen Ansatz zu diesem Problem lieferte R. Kalman im Jahr 1960 mit dem, nach ihm benannten, Kalman Filter \cite{kalman}.
\\Ungeachtet des Namens handelt es sich beim Kalman Filter eher um einen Schätzwertalgorithmus, als um einen Filter. Der Algorithmus basiert auf Ansätzen der Wahrscheinlichkeitstheorie nach Bayes. Ermittelt werden die optimalen Schätzwerte auf Grundlage des Wissens über deterministische und statistische Systemeigenschaften. Ausgehend von einem festgelegten Satz an Schätzwerten, werden die Zustandswerte rekursiv, als gewichteter Durchschnitt der vorheringen Werte und neuen Messwerten berechnet. Da in jedem Iterationsschritt des Verfahrens lediglich die aktuellen Messdaten benötigt werden, eignet sich das Kalman Filter gut für Echtzeitanwendungen. \cite{P32}
\\Das Modell des Kalman Filters besteht aus fünf Hauptelementen: dem Zustandsvektor, dem Systemmodell, dem Messvektor, dem Messungsmodell und dem Filter Algorithmus.
Der \textit{Zustandsvektor} $\mathbf{x}$, oft auch state vector genannt, ist der Parametersatz durch welchen das System beschrieben wird. Er beeinhaltet die Zustände, welche durch das Kalman Filter geschätzt werden. Eng mit dem Zustandsvektor verbunden, ist die \textit{Fehlerkovarianzmatrix} $\mathbf{P}$ (englisch: error covariance matrix). Durch die Diagonaleinträge der Matrix werden die Varianzen der geschätzten Parameterwerte dargestellt. Die restlichen Einträge der Matrix repräsentieren die Korrelationen zwischen den Fehlern der Zustandsschätzungen. Sowohl der Zustandsvektor als auch die Fehlerkovarianzmatrix müssen zu Beginn vom Nutzer vorgegeben werden. Die Startwerte der Matrix geben an wie sehr den verschiedenen Schätzwerten getraut wird. Mathematisch lässt sich die Fehlerkovarianzmatrix darstellen als
\begin{equation}\label{P}
  \mathbf{P} = E((\mathbf{x} - \mathbf{\hat x})(\mathbf{x} - \mathbf{\hat x})^T)   .
\end{equation}
Durch das \textit{Systemmodell} werden die zeitlichen Veränderungen der Zustände und der Fehlerkovarianzmatrix beschrieben. Da lediglich auf bekannte Systemeigenschaften zurückgegriffen wird, handelt es sich hierbei um ein deterministisches Modell.
Der \textit{Messvektor} $\mathbf{z}$ (engl.: measurement vector) umfasst einen Satz aus Simultanmessungen der Systemeigenschaften, als Funktionen des Zustandsvektors. Er besteht aus einer deterministischen, vom Zustandsvektor abhängigen, Funktion und einem Rauschterm $\mathbf{v}_m$. Ähnlich der Fehlerkovarianzmatrix in Bezug zum Zustandsvektor, besteht eine Relation zwischen dem Messvektor und der \textit{Kovarianzmatrix des Messrauschens} $\mathbf{R}$ (engl.: measurement noise covarince matrix). Diese beschreibt die Rauschverteilung der Messwerte und wird mathematisch dargestellt durch
\begin{equation}\label{R}
  \mathbf{R} = E(\mathbf{v}_m \mathbf{v}_m^T)   .
\end{equation}
 Die diagonalen Einträge sind durch die Varianzen der Messwerte gegeben, während die restlichen Einträge die Korrelationen zwischen den Werten darstellen. Da die Komponenten meist unabhängig voneinander sind, handelt es sich bei der Kovarianzmatrix des Messrauschens in den meisten Fällen um eine Diagonalmatrix.
Durch das \textit{Messungsmodell} (engl.: measurement model) werden die Veränderungen des Messvektors als Funktion des Zustandsvektors ohne Einbezug des Messrauschens dargestellt.
Der \textit{Filter Algorithmus} nutzt dann den Messvektor, das Messungs- und das Systemmodell um die optimalen Schätzwerte des Zustandsvektors zu ermitteln. \cite{P32}
\\Der Algorithmus arbeitet in zwei Schritten, dem Zeitupdate und dem Messungsupdate. Das \textit{Zeitupdate} (engl.: time-update) dient dazu, erfahrungsunabhängige Schätzwerte des Zustandsvektors und der Fehlerkovarianzmatrix bis zum nächsten Messzeitpunkt zu propagieren. Aus diesem Grund wird das Zeitupdate häufig als der Prädiktionsschritt des Filters bezeichnet.
Das \textit{Messungsupdate} (engl.: measurement-update) stellt die Rückkopplung des Algorithmus dar. In dieser, auch als Korrekturschritt bezeichneten, Phase des Algorithmus, werden die propagierten Schätzwerte durch die aktuellen Messwerte angepasst. \cite{P24}
\\Um die Funktionsweise des Algorithmus näher zu erläutern wird von einem durch
\begin{equation}\label{xk}
  \mathbf{x}_k = \mathbf{Ax}_{k-1} + \mathbf{Bu}_k + \mathbf{w}_{m}
\end{equation}
beschriebenen System ausgegangen. Hierbei stellt \textbf{A} die Übergangsmatrix und $\mathbf{w}_m$ das Systemrauschen dar. Die Matrix \textbf{B} verknüpft den optionalen Eingangsvektor \textbf{u} mit dem Zustandsvektor. Dieser Term entfällt jedoch in vielen Berechnungen. Der Messvektor wird dargestellt durch
\begin{equation}\label{zk}
  \mathbf{z}_k = \mathbf{H}\mathbf{x}_k+\mathbf{v}_m .
\end{equation}
$\mathbf{H}$ bezeichnet die Messungsmatrix, durch welche die Veränderungen des Messvektors abhängig vom Zustandvektor beschrieben werden. $\mathbf{v}_m$ ist der Ausdruck für das Messrauschen. Die beiden Rauschgrößen $\mathbf{w}_m$ und $\mathbf{v}_m$ werden als voneinander unabhängige, weiße und normalverteilte Zufallsgrößen angenommen. \cite{P24}
\\Das Zeitupdate erfolgt in vier Schritten. Im Ersten erfolgt gegebenfalls die Berechnung der Übergangmatrix $\mathbf{A}_{k-1}$. Da diese jedoch häufig als konstant angenommen werden kann, entfällt dieser Berechnungsschritt mitunter. Die Übergangsmatrix beschreibt wie sich die einzelnen Zustände als Funktion der Systemdynamik mit der Zeit ändern. Somit ist Sie immer eine Funktion des Zeitintervalls zwischen zwei Filteriterationen. Auch der zweite Schritt, die Berechnung der Kovarianzmatrix des Systemrauschens $\mathbf{Q}_{k-1}$ entfällt häufig, da die Matrix als konstant angenommen werden kann. Sie dient der Beschreibung des Anstiegs der Abweichungen der Zustandsschätzungen, aufgrund des Systemrauschens und ist formal gegeben durch
\begin{equation}\label{Q}
  \mathbf{Q} = E(\mathbf{w}_m \mathbf{w}_m^T).
\end{equation}
Die Prädiktion des Zustansvektors, in Form von
\begin{equation}\label{xk-}
  {\mathbf{\hat x}_{k}}^- = \mathbf{A}_{k-1}{\mathbf{\hat x}_{k-1}}^+
\end{equation}
erfolgt als dritter Schritt des Zeitupdates. Dabei gibt $ {\mathbf{\hat x}_{k}}^- $ den Wert des geschätzten Zustandsvektor nach dem \textit{Zeitupdate} im k-ten Iterationsschritt an, während ${\mathbf{\hat x}_{k-1}}^+$ den Wert des geschätzten Zustandsvektors nach dem \textit{Messungsupdate} im (k-1)-ten Integrationsschritt darstellt. Diese Notiation unter Verwendung des hochgestellten - und + findet auch in allen nachfolgenden Gleichungen Anwendung. Im abschließenden, vierten Schritt des Zeitupdates wird die Fehlerkovarianzmatrix für den aktuellen Iterationsschritt propagiert. Dies erfolgt über
\begin{equation}\label{Pk-}
  {\mathbf{P}_{k}}^- = \mathbf{A}_{k-1}{\mathbf{P}_{k-1}}^+{\mathbf{A}_{k-1}}^T + \mathbf{Q}_{k-1} .
\end{equation}
Das Messungsupdate besteht aus insgesamt fünf Berechnungen. Zunächst erfolgt die Berechnung der Kovarianzmatrix des Messrauschens zu
\begin{equation}\label{Rk}
  \mathbf{R}_{k} = E((\mathbf{z}_{k} - \mathbf{Hx}_k)(\mathbf{z}_{k} - \mathbf{Hx}_k)^T).
\end{equation}
In einigen Fällen, kann auch diese Matrix als konstant angenommen werden. Jedoch ist in diesem Fällen, darauf zu achten, dass die Kovarianzmatrix die nachfolgend durch
\begin{equation}\label{K}
  \mathbf{K}_k = {\mathbf{P}_k}^- \mathbf{H}^T(\mathbf{H}{\mathbf{P}_k}^-\mathbf{H}^T + \mathbf{R}_k)^{-1}
\end{equation}
berechnete Kalmanverstärkung direkt beeinflusst. \cite{P32}
Die Kalmanverstärkung (engl.: Kalman Gain) dient der Gewichtung der Messinformationen. Wird sie zu klein gewählt ist es möglich, dass die Schätzwerte des Zustandsvektors langsamer als nötig gegen den angestrebten Wert konvergieren. Desweiteren kann nur langsam auf Änderungen im System eingangen werden. Somit werden in diesem Fall die propagierten Werte stärker gewichtet als die Gemessenen.  Bei einer sehr großen Kalmanverstärkung ist es möglich, dass das Filter nicht mehr im Gleichgewicht arbeitet und somit die tatsächlichen Messwerte des Messvektors $\mathbf{z}_k$ den Filterprozess stärker beeinflussen. \cite{P24}
Darauffolgend wird mittels Gl.~(\ref{zk}) der Wert des Messvektors aktualisiert.
Abschließend erfolgen die Korrekturen des Zustandsvektors und der Fehlerkovarianzmatrix über
\begin{equation}\label{xk+}
  {\mathbf{\hat x}_k}^+ = {\mathbf{\hat x}_k}^- + \mathbf{K}_k(\mathbf{z}_k - {\mathbf{H\hat x}_k}^-)
\end{equation}
und
\begin{equation}\label{Pk+}
  {\mathbf{P}_k}^+ = (\mathbf{I} - \mathbf{K}_k\mathbf{H}){\mathbf{P}_k}^-.
\end{equation}
\cite{P32}
\\Da das Kalman Filter auf die Zustandsschätzung linearer Systeme beschränkt ist wurden über die Jahre, auf ihm basierend, einige Ansätze zur Zustandsschätzung nichtlinearer Systeme entwickelt.
Einer dieser Ansätze ist das \textit{Extended Kalman Filter} (EKF). Bei diesem Verfahren wird eine Gaußverteilung der Zustände angenommen. Das EKF beruht darauf, das nichtlineare System einer Linearisierung erster Ordnung zu unterziehen, um zur weiteren Berechnung die Gleichungen des Kalman Filters nutzen zu können.
Durch die Linearisierung kann es jedoch zu erheblichen Fehlern in Mittelwert und Kovarianz des Systems kommen, was mitunter in stark divergierendem Verhalten des Filters resultiert. Aus diesem Grund eigenet sich die Anwendung des EKF lediglich für Systeme mit geringer Nichtlinearität.
Desweiteren ist es im Zuge der Linearisierung nötig Jacobi-Matrizen zu berechnen, was einen erheblichen Rechenaufwand mit sich bringt. \cite{P36}
\\Um auch die Problematiken des EKF reduzieren zu können, wurde das \textit{Unscented Kalman Filter} entwickelt.

\subsubsection{Unscented Kalman Filter}
Das Unscented Kalman Filter wurde erstmals von S. J. Julier, J. K. Uhlmann und H. F. Durrant-Whyte vorgestellt \cite{P3,P3-2}. Ähnlich dem EKF, wird auch beim UKF von einer gaußschen Verteilung der Zustände ausgegangen. Diese wird über einen Satz diskreter Abtastpunkte, welche als Sigma Punkte $\mathbf{\mathcal{X}}_i$ bezeichnet werden, festgelegt. Mithilfe dieser Punkte ist eine Parametrisierung des Mittelwertes und der Kovarianz möglich. Die Sigma Punkte werden genutzt um anschließend eine nichtlineare Transformation, welche als \textit{Unscented Transformation} (UT) bezeichnet wird, durchführen zu können. Die Möglichkeit dieser nichtlinearen Transformation führt dazu, dass die, beim EKF notwendige, Berechnung der Jacobi-Matrizen entfällt. \cite{P36}
\\Zur präzisen, mathematischen Beschreibung des UKF wird von einem  Systemmodell der Form
\begin{equation}\label{sysmodell_xk+1}
  \mathbf{x}_{k+1} = \mathbf{f}(\mathbf{x}_k,\mathbf{u}_k) + \mathbf{w}_k
\end{equation}
\begin{equation}\label{sysmodell_zk}
  \mathbf{z}_{k} = \mathbf{h}(\mathbf{x}_k) + \mathbf{v}_k
\end{equation}
ausgegangen. Hierbei stellt $\mathbf{x} \in \mathbb{R}^{N_x}$ den Zustandsvektor, $\mathbf{f} : \mathbb{R}^{N_x} \times \mathbb{R}^{N_u} \to \mathbb{R}^{N_x}$ und $\mathbf{h} : \mathbb{R}^{N_x} \to \mathbb{R}^{N_x}$ nichtlineare Vektorfelder,
$\mathbf{u} \in \mathbb{R}^{N_u}$ den Eingangsvektor und $\mathbf{z} \in \mathbb{R}^{N_z}$ den Messvektor dar. $N_u$, $N_x$ und $N_z$ stehen dabei jeweils für die Anzahl der Einträge der Vektoren $\mathbf{u,x}$ und $\mathbf{z}$.
Für die Rauschterme $\mathbf{w}_k$ und $\mathbf{v}_k$ wird wie beim Kalman Filter angenommen, dass es sich um unkorrelierte, mittelwertfreie, weiße Gaußprozesse handelt. Die Kovarianzmatrizen der Rauschterme sind über Gl.~(\ref{R}) und Gl.~(\ref{Q}) definiert.
Angenommen $\mathbf{\hat x}_k$ und $\mathbf{P}_k$ seien initialisiert als
\begin{equation}\label{x0}
  \mathbf{\hat x}_0 = E (\mathbf{x}_0)
\end{equation}
und
\begin{equation}\label{P0}
  \mathbf{P}_0 = E((\mathbf{x_0 - \hat x_0})(\mathbf{x_0 - \hat x_0})^T) .
\end{equation}
Dann werden die $(2N_n+1)$ Sigma-Punkte berechnet zu
\begin{equation}\label{Sigmas}
  \mathbf{\mathcal{X}}_i =
  \begin{cases}
    \mathbf{\hat x}_k & , i=0 \\
    \mathbf{\hat x}_k + (\gamma \cdot \sqrt {P_k})_i & , i = 1,2,...,N_x \\
    \mathbf{\hat x}_k - (\gamma \cdot \sqrt {P_k})_{i-N_x} & , i=(N_x+1),...,2N_x.
  \end{cases}
\end{equation}
Hierbei bezeichnet $(...)_i$ die i.te Spalte der Matrix $\mathbf{P}$. Die Variable $\gamma$ ist gegeben durch
\begin{equation}\label{c}
  \gamma = \sqrt {N_x+\lambda}  \text{ mit } \lambda = \alpha^2(N_x+\kappa)-N_x
\end{equation}
, wobei $\lambda$ und $\kappa$ Skalierungsparameter darstellen. In den meisten Fällen wird  $\kappa = 3 - N_x$ gewählt. Der Paramter $\alpha$ gibt die Ausbreitung der Sigma-Punkte um $\mathbf{\hat x}$ an und wird meist in einem Intervall von $[10^{-4},1]$ gewählt. Für die Berechnung der Wurzel der Fehlerkovarianzmatrix in (\ref{Sigmas}) wird auf die Cholesky-Zerlegung zurückgegriffen, nach der
\begin{equation}\label{chol}
    chol(\mathbf{P}_k) = \sqrt {\mathbf{P}_k}
\end{equation}
gilt.
Wie auch beim Kalman Filter, sind Berechnungen des UKF unterteilt in Zeit- und Messungsupdate.
Innerhalb des Zeitupdates werden die zuvor berechneten Sigma-Punkte zunächst unter Verwendung des dynamischen Systemmodells propaiert zu:
\begin{equation}\label{prop.sigma1}
  \mathbf{\tilde {\mathcal{X}}}_k = \mathbf{f}(\mathbf{\mathcal{X}}_{k-1},\mathbf{u}_{k-1}) .
\end{equation}
Der geschätzte Zustandsvektor wird somit berechnet zu
\begin{equation}\label{xk-ukf}
  {\mathbf{\hat x}_{k}}^- = \sum_{i=0}^{2N_x} {{W_i}^{(m)}\mathbf{\tilde {\mathcal{X}}}_{i,k}}
\end{equation}
, wobei ${W_i}^{(m)}$ die Gewichtung der jeweiligen Sigma-Punkte durch
\begin{equation}\label{Wim}
    {W_i}^{(m)} =
    \begin{cases}
        \frac{\lambda} {(N_x+\lambda)} & , i=0 \\
        \frac{1} {2(N_x+\lambda)} & , i=1,2,...,2N_x
    \end{cases}
\end{equation}
angibt. Die Fehlerkovarianzmatrix des Zeitupdateschrittes wird berechnet durch
\begin{equation}\label{Pk-ukf}
  {\mathbf{P}_k}^- = \sum_{i=0}^{2N_x} {{W_i}^{(c)}(\mathbf{\tilde {\mathcal{X}}}_{i,k} - {\mathbf{\hat x}_k}^-){(\mathbf{\tilde {\mathcal{X}}}_{i,k} - {\mathbf{\hat x}_k}^-)}^T} + \mathbf{Q}
\end{equation}
, mit Gewichtung ${W_i}^{(c)}$ gegeben als
\begin{equation}\label{Wic}
  {W_i}^{(c)} =
  \begin{cases}
      \frac{\lambda} {(N_x+\lambda)} + (1-\alpha^2+\beta) & , i=0 \\
      \frac{1} {2(N_x+\lambda)} & , i=1,2,...,2N_x
  \end{cases} .
\end{equation}
Der Parameter $\beta$ dient dazu, die UT an die angenommene Zustandsverteilung anzupassen. Für die hier gewählte Gaußverteilung ist eine Wahl von $\beta = 2$ als optimal anzusehen.
Nun werden mittels (\ref{Sigmas}) erneut Sigma-Punkte $\mathbf{\mathcal{X}_k}^-$, unter Verwendung des geschätzten Zustandsvektors ${\mathbf{\hat x}_k}^-$ und der propagierten Fehlerkovarianzmatrix ${\mathbf{P}_k}^-$, generiert.
\\Den Übergang zum Messupdate bildet die Schätzung des Messvektor mittels
\begin{equation}\label{zk_hat-}
    {\mathbf{\hat z}_k}^- = \sum_{i=0}^{2N_x} {{W_i}^{(m)}\mathbf{Z}_{i,k}}
\end{equation}
, wobei durch
\begin{equation}\label{Zk}
    \mathbf{Z}_k = \mathbf{h}(\mathbf{\mathcal{X}_k}^-)
\end{equation}
die zuvor generierten Sigma-Punkte, unter Gewichtung, mit in die Berechnung einbezogen werden.
Die Berechnungen des Messupdates umfassen unter Anderem die Berechnung der Kalman Verstärkung
\begin{equation}\label{Kukf}
    \mathbf{K}_k = \mathbf{P}_{xz}{\mathbf{P}_{zz}}^{-1}
\end{equation}
mit
\begin{equation}\label{Pzz}
  \mathbf{P}_{zz} = \sum_{i=0}^{2N_x} {{W_i}^{(c)}(\mathbf{Z}_{i,k}-{\mathbf{\hat z}_k}^-)(\mathbf{Z}_{i,k}-{\mathbf{\hat z}_k}^-)^T} + R
\end{equation}
und
\begin{equation}\label{Pxz}
  \mathbf{P}_{xz} = \sum_{i=0}^{2N_x} {{W_i}^{(c)}(\mathbf{\mathcal{X}_{i,k}}^--{\mathbf{\hat x}_k}^-)(\mathbf{Z}_{i,k}-{\mathbf{\hat z}_k}^-)^T}}.
\end{equation}
Unter Verwendung der Kalman Verstärkung werden abschließend der geschätzte Zustandsvektor $\mathbf{\hat x}_k$ und die Fehlerkovarianzmatrix $\mathbf{P}_k$ aktualisiert zu
\begin{equation}\label{xhatk+}
  {\mathbf{\hat x}_k}^+ = {\mathbf{\hat x}_k}^- + \mathbf{K}_k(\mathbf{z}_k-{\mathbf{\hat z}_k}^-)
\end{equation}
und
\begin{equation}\label{Pk+ukf}
  {\mathbf{P}_k}^+ = {\mathbf{P}_k}^- - \mathbf{K}_k\mathbf{P}_{zz}{\mathbf{K}_k}^T
\end{equation}


\section{Bewegungserfassung}
Bei der Bewegungserfassung, welche häufig auch als \textit{Motion Capturing} bezeichnet wird, erfolgt eine Aufzeichnung und Umwandlung realer Bewegungen zu einer digitalen, für Computer erfassbare, Darstellung eben dieser.
Dies wird durch das Verfolgen markanter Punkte, auch \textit{tracken} genannt, während der Bewegung eines Körper und anschließender Erstellung eines 3D-Modells zur weiteren Verarbeitung.
Motion Capturing kann unter Verwendung verschiedener Systeme realisiert werden. Die größte Verwendung von Motion Capturing Systemen liegt in Bereichen wie der Medizin oder auch dem Entertainment, z.B. in Form der Animation von Filmcharakteren. \cite{mocap_def}

\subsection{Definitionen}
Zum besseren Verständnis der Thematik des Motion Capturing werden an dieser Stelle einige grundlegende Begriffe erläutert.
\subsubsection{Aktionsbereich}
Hierunter wird der Bereich verstanden, in welchem der Tracker, unter Einhaltung der angegebenen Auflösung und Genauigkeit, zuverlässige Ergebnisse liefern kann \cite{P25}.
Bei optischen Systemen wird hiermit der Raum beschrieben, in welchem die Bildausschnitte der Kameras eine Bestimmung der Position des Markers zulassen \cite{mocap}.
\subsubsection{Auflösung}
Unter der Auflösung eines Trackingsystems versteht man die kleinste Änderung in Position und Orientierung, welche durch das System aufgezeichnet werden kann \cite{P25}.
\subsubsection{Genauigkeit}
Bei der Genauigkeit handelt es sich um den Messfehler, welcher während der Messung von Orientierung und Position auftritt \cite{P25}.
\subsubsection{Latenzzeit}
Die Latenzzeit beschreibt die Verzögerung zwischen der tatsächlichen Änderung in Position und Orientierung und dem Zeitpunkt zu welchem das System die Änderung erkennt \cite{vr}.
\subsubsection{Marker}
Bei Markern handelt es sich um Hilfsmittel zur Kennzeichnung markanter Punkte am zu trackenden Objekt. Hierbei wird zwischen aktiven und passiven Markern unterschieden.
Unter aktiven Markern versteht man solche, welche, meist in Form vom LEDs, infrarotes Licht emittieren. Dieses wird von speziellen, mit den Markern synchronsierten, Kameras erfasst. \cite{ART}
\\Aktive Marker eignen sich besonders für den Einsatz in Verbindung mit relexionsreichen Materialien und bei staubigen Umgebungsverhältnissen \cite{qualsys_act}.
\\Im Gegensatz zu den aktiven Markern sind passive Marker retroreflektiv. Anstatt wie aktive Marker infrarotes Licht zu emittieren, reflektieren sie einfallende Infrarotstrahlung in die Einfallsrichtung gebündelt. \cite{ART}
\subsubsection{Time of Flight}
Unter der Time of Flight (ToF) versteht man die Zeit, welche ein Signal von seiner Quelle bis zum messenden Sensor benötigt. Mithilfe der ToF kann aufgrund der Proportionalität der Dauer zur Strecke eine Aussage über die zurückgelegte Strecke getroffen werden. \cite{P24}
\subsubsection{Updaterate}
Die Updaterate eines Systems beschreibt die Anzahl der Abtastungen eines Sensors pro Sekunde. Je höher die Updaterate ist, desto besser kann das dynamische Systemverhalten beschrieben werden. \cite{vr}
\subsubsection{Verdeckung}
Verdeckung beschreibt den Zustand bei welchem ein Marker von einer oder mehreren Kameras nicht mehr wahrgenommen werden kann, weil Teile des zu trackenden Objekts oder andere Gegenstände die Sicht auf den Marker verhindern \cite{occlusion}.

\subsection{Arten der Aufnahmesysteme}
Im Bereich der Bewegungserfassung wird zwischen drei verschiedenen Gruppen der Aufnahmesysteme unterschieden. Hierbei erfolgt die Zuordnung über die Anbringungsposition der Sensoren und Datenquellen. \cite{P25}
\subsubsection{Outside-In Systeme}
Das Prinzip der Outside-In Systeme sieht eine Anbringung der Sensoren an festen Positionen im Raum vor. Die Sensoren werden hierbei auf das zu trackende Objekt ausgerichtet. Dieses ist wiederum mit Markern versehen, über welche die Datenerfassung erfolgen kann.
Hierbei ist auf eine ausreichend hohe Anzahl von Sensoren zu achten, um eine möglichst fehlerfreie Auswertung der Aufzeichnungen garantieren zu können. \cite{P25}
\\Dieses Verfahren findet Anwendung in optischen Trackingsystem, welche darauf beruhen, dass Bewegungen des, mit Markern versehenen Objektes, über Kameras erfasst werden \cite{optsys2}.
\subsubsection{Inside-In Systeme}
Bei Inside-In Systemen, werden nicht nur die Datenquellen am Aktuer platziert, sondern auch die Sensoren.
Ein klassisches Beispiel hierfür stellen Exo-Skelett Systeme dar, bei welchen die relativen Bewegungen des Darstellers über Potentiometer bestimmt werden können.
Ein wesentlicher Vorteil dieser Systeme, gegenüber den Outside-In Systemen, besteht darin, dass das Auftreten von Verdeckungsproblemen vermieden wird und somit das Tracken mehrerer Akteure zum gleichen Zeitpunkt, ohne die Gefahr eines Datenverlustes möglich ist.
Oft wird hierbei jedoch die Bewegungsfreiheit der Darsteller durch das mechanische Skelett stark eingeschränkt. \cite{optsys2}
\subsubsection{Inside-Out Systeme}
Inside-Out Systeme nutzen am Darsteller positionierte Sensoren, welche Signale von fest in der Umgebung platzierten Datenquellen empfangen und auswerten können \cite{P25}.
Ein Beispiel für Systeme dieser Art sind magnetische Trackingsysteme. Hierbei wird durch externe Quellen ein Magnetfeld erzeugt, dessen Feldstärke dann an den jeweiligen Sensoren gemessen wird, um eine Entfernung zur Quelle zu bestimmen. Über die relativen Positionen der Sensoren zu verschiedenen Quellen ist anschließend eine räumliche Positionsbestimmung möglich. \cite{optsys2}

\subsection{Optische Trackingsysteme}
Unter optischen Trackingsystemen werden all die Verfahren des Motion Capturing zusammengefasst, welche auf einer Bewegungserfassung unter Verwendung von Lichtquellen und optischen Sensoren basieren. \cite{P31}
Im Regelfall werden für ein präzises optisches Tracken, je nach verwendetem System, vier bis zwölf Spezialkameras benötigt.\cite{optsys1}
\\Das Themengebiet der optischen Trackingsysteme lässt sich in das markerbasierte und das markerlose Tracking unterteilen.
\\Beim markerbasierten Tracking wird zwischen der Verwendung von aktiven und passiven Markern unterschieden \cite{P25}. Die Funktionsweise beider Teilbereiche beruht jedoch darauf, dass Markerpositionen mithilfe von Spezialkameras erfasst werden. Hierbei werden die Kameras meist am oberen Rand der Aufnahmeumgebung, in regelmäßigen Abständen zueinander, angebracht.
Eine typischerweise beim passiven Tracking verwendete Kamera ist so aufgebaut, dass ringförmig um das Objektiv herum, eine Vielzahl von Infrarot-Leuchtdioden angebracht ist, über welche die passiven Marker angestrahlt und die Reflexion anschließend vom Objektiv erfasst werden kann. \cite{optsys2}
Ein Beispiel hierfür, stellen die in dieser Arbeit, im Referenzsystem verwendeten Oqus Kameras der Firma Qualisys dar \cite{oqus}.
\\Je nach Art und Aufbau des verwendeten Systems kann es zu großen Schwankungen bezüglich des Aktionsbereichs zwischen den einzelnen Systeme kommen \cite{P24}. Bei geeigneter Wahl des Systems ist das erzielen sehr großer Aktionsbereiche jedoch durchaus möglich \cite{P25}.
Ein weiterer positiver Aspekt des markerbasierten Trackings liegt in der leichten Erweiterbarkeit des Messaufbaus. So können ohne großen Aufwand zusätzliche Marker am Akteur angebracht werden, ohne dass dieser wesentlich in seiner Bewegungsfreiheit beeinträchtig wird. \cite{optsys2}
\\Das markerbasierte Tracking bietet jedoch nicht nur Vorteile. Ein wesentlicher Nachteil Systeme dieser Art ist durch die Verdeckung gegeben. So kann es vorkommen das, entweder vom Darsteller selbst, durch weitere Akteure oder aber auch durch Gegenstände innerhalb des Aktionsbereichs, ein Marker verdeckt wird und so nicht von ausreichend Kameras eingesehen werden kann. Ein Datenverlust aufgrund eines aus dem Bildbereich einer Kamera entwichenen Markers kann teilweise durch Interpolationsverfahren bei der späteren Datenverarbeitung behoben werden. \cite{optsys2} Um die Problematik der Verdeckung zu reduzieren, bevor es zu Datenverlusten kommen kann, ist es mitunter sinnvoll weitere Kameras in die Aufnahmeumgebung zu integrieren \cite{P25}.
\\Beim markerlosen Tracking erfolgt die Bewegungserfassung, abgesehen von Kameras, ohne weitere Hilfsmittel. Ein Beispiel hierfür sind ToF-Kameras. Diese bestimmen die Position es zu trackenden Objektes über eine Lichtzeitmessungen. Hierbei wird die Zeit bestimmt, welche ein, von der Kamera ausgestrahlter, Lichtimpuls benötigt um nach Reflexion wieder die Kamera zu erreichen. \cite{basler}
Ein weiteres markerloses Trackingverfahren ist das sogenannte Laser-Ranging. Hierbei wird ein Beugungsgitter zwischen Objekt und Laser platziert und durch den Laser angestrahlt, wodurch ein Intereferenzmuster auf dem Objekt erzeugt wird. Dieses Interferenzmuster wird anschließend mithilfe von Kameras gescannt, welche die Intensität des Musters bestimmen. Hieraus wird dann der Contrast Ratio bestimmt, welcher dazu dient die Distanz zwischen getracktem Objekt und Beugungsgitter zu bestimmen. Ein Nachteil dieses Verfahrens ist der stark eingeschränkte Aktionsbereich, welcher daher rührt, dass mit steigender Entfernung zwischen Beugungsgitter und Objekt die Messgenauigkeit stark abnimmt und somit keine qualitativ wertvolle Aussage über die tatsächliche Distanz mehr getroffen werden kann. \cite{P25}
\\Unabhängig davon, ob es sich um markerbasiertes oder markerloses Tracking handelt zeichnen sich optische Verfahren zur Bewegungserfassung besonders, durch das Erzielen sehr hoher Genauigkeit und Auflösung innerhalb ihrer jeweiligen Aktiobsbereiche aus \cite{P24}. Auch die Verfügbarkeit hoher Updateraten stellt in beiden Fällen einen positiven Aspekt der Systeme dar \cite{P25}.
Der häufig hohe Rechenaufwand, welcher durch eventuelle Nachbearbeitung der Daten entsteht ist ein wesentlicher Nachteil dieser Art der Bewegungserfassung \cite{optsys1}. Da es bei optischen Verfahren zu, durch Umgebungslicht verursachten, Störungen kommen kann ist ein erfolgreicher Einsatz dieser Techniken im wesentlichen nur in Messlaboren unter konstanten Umgebungsverhätnissen möglich \cite{P25}.

\subsection{Tracking mit Intertialsensorik}
In Trackingsystemen mit Inertialsensorik kommen sowohl Beschleunigungssensoren, auch Accelerometer genannt, als auch Gyroskope zum Einsatz. Dabei dienen die Beschleunigungssensoren der Bestimmung der Position des Objektes, während mithilfe der Gyroskope eine Aussage über die Objektorientierung getroffen werden kann. \cite{P24}
Hierbei beruhen alle Bestimmungen auf dem Messen von Trägheitskräften \cite{P32} unter der Anwendung des 2. Gesetzes von Newton, in der Form
\begin{equation}\label{Newton2.1}
    F = ma
\end{equation} oder
\begin{equation}\label{Newton2.2.}
   Q = I\alpha
\end{equation} \cite{P24}.
\\Da sowohl die Messungen der Beschleunigungssensoren, als auch die der Gyroskope jeweils nur eine Achse einschließen, müssen für die Ausführung dreidimensionaler Messungen jeweils drei Accelerometer und drei Gyroskope zu einer Inertial measurement unit (IMU) kombiniert werden \cite{P32}. Hierfür werden die Sensoren einer Art jeweils orthogonal zueinander auf einer Platform aufgeordnet \cite{P24}.
\\Die am häufigsten genutzte Art der Accelerometer sind die Pendelbeschleunigungsmesser. Bei diesen ist eine bekannte Masse an einer Seite einer gedämpften Feder befestigt, während die andere Seite am Gehäuse des Sensors angebracht ist. \cite{P32}
Während keine Beschleunigung auf den Sensor wirkt, ruhen Feder und Masse. Das Sensorgehäuse und die Referenzmasse sind somit ohne Versatz zueinander ausgerichtet. Sobald jedoch eine Kraft auf das Gehäuse einwirkt und dieses einer Beschleunigung ausgesetzt wird entsteht, aufgrund der Trägheit der Masse, ein Versatz.
Dieser Versatz und die Stauchung bzw. Streckung der Feder können als zur Beschleunigung proportional angenommen werden. Mithilfe eines Wandler, welcher entweder potentiometrisch oder piezoelektrisch sein kann, wird aus dem Versatz ein Signal abgeleitet. Handelt es sich bei dem Wandler um ein Potentiometer, so ist der Versatz der Masse mit dem Regler des Potentiometers gekoppelt. Wird jedoch ein piezoelektrischer Wandler genutzt, erzeugt der verwendete piezokristall eine elektrische Ladung, wenn, durch den Versatz, eine Kraft auf ihn ausgeübt wird.
Da es nicht möglich ist die Beschleunigung direkt zu messen, wird unter Beachtung von Gl. \ref{Newton2.1}, die auf die Masse ausgeübte, zur Beschleunigung proportionale Kraft gemessen. Über die Relation
\begin{equation}\label{beschl}
  a = \frac{d^2r}{dt^2}
\end{equation}
kann die Position r durch zweifache Integration von \textit{a} bestimmt werden, zu
\begin{equation}\label{pos}
  r = \iint{a dt^2}.
\end{equation}
Die Einwirkung der Gravitation auf die Masse muss vor der Berechnung eliminiert werden. \cite{P24}
\\Die Vibrating Beam Accelerometer (VBA), oder auch Resonanzbeschleuningungssensoren, stellen eine Alternative zu den Pendelbeschleunigungsmessern dar. Ihr Aufbau ähnelt denen der Pendelbeschleunigungssensoren, jedoch wird die Masse zusätzlich durch einen Schwingbalken entlang der Messachse verstärkt. Dieser Balken drückt oder zieht, je nach Belastung, an der Masse, wodurch es zu einer Streckung oder Stauchung des Balkens kommt.
Die Resonanzfrequenz, mit welcher der Balken schwingt, verändert sich abhängig von der Streckung/Stauchung. Misst man diese Frequenzänderung so kann eine Aussage über die einwirkende Kraft und somit die Beschleunigung getroffen werden. \cite{P32}
\\Gyroskope dienen der Messung der Winkelrate \cite{P32} und folgen dem Prinzip der Drehimpulserhaltung \cite{P24}. Hauptsächlich wird von sogenannten Spinning Mass Gyroskopen Gebrauch gemacht. Diese basieren auf dem Gedanken, dass auf eine rotierende Masse ein Drehmoment erzeugt wird. Durch dieses präzediert die Rotationsachse im rechten Winkel zu sich selbst und zur Achse des erzeugten Drehmomentes. Ist die rotierende Masse kardanisch, also in zwei sich schneidenden, zueinander rechtwinkligen Drehlagern aufgehangen, so ist es möglich die Richtungsveränderung zu messen. Der Winkel zwischen dem Gyroskop und dem Gehäuse ist hierbei die Maßeinheit für den Drehimpuls bzw. die Winkelgeschwindigkeit dar. \cite{P24}
Optische und vibrierende Gyroskope sind mögliche Alternativen zu Spinning Mass Systemen, welche jedoch weitaus weniger verbreitet sind \cite{P32}.
\\Vor einigen Jahren lag ein großer Nachteil der Inertialsensorik in der Größe der Sensoren. Seit der Entwicklung von \textit{microelectronic mechanical systems} (MEMS) ist diese Problematik erheblich gesunken und ein Tracken menschlicher Bewegungen mittels inertialer Sensoren ermöglicht worden. \cite{P31}
Ein entscheidender Nachteil, welcher aber auch nach der Entwicklung der MEMS bestehen bleibt ist, dass inertiale Systeme, durch die Erforderlichkeit von Integration zur Berechnung von Position und Orientierung, sehr anfällig für das Auftreten von Drifts sind. Mit steigendem Messzeitraum divergieren Fehler in Position und Orientierung. Dieser Problematik kann lediglich durch ein regelmäßiges neu kalibrieren des Systems oder den Vergleich mit anderen Trackingmethoden entgegengewirkt werden. Wird nur über kurze Zeitverläufe und mit hohen Frequenzen gemessen kann allerding eine hohe Performance erzählt werden. \cite{P24} Da unter Verwendung von Intertialsensorik die Problematik der Verdeckung behoben werden kann, stellen sie eine gute Ergänzung zu optischen Trackingsystemen dar. Desweiteren reagieren Inertialsensoren nicht empfindlich auf äußere elektromagnetische Felder und sind resistent gegenüber Umgebungsrauschen. Eine Nutzung mit hohen Updateraten und der Vorteil der geringen Latenz dieser Art der Trackingsysteme sind weiter positive Aspekte der Inertialsensorik. \cite{P31}

\section{Body Sensor Network}
Eine weitere Art der Datenerhebung für die Positionsbestimmung ist durch sogenannte \textit{Body Sensor Networks} (BSN) gegeben.
Bei diesen handelt es sich um Sensornetzwerke, welche meist aus mehreren Sensorknoten (SN) und einem Masterknoten (MN) bestehen.
Innerhalb dieser SN wird überwiegend auf das Tracking mit Inertialsensorik zurückgegriffen.
Kommunikation innerhalb des BSN ist sowohl zwischen den SN untereinander, als auch zwischen einzelnen SN und dem MN möglich.
Dadurch, dass die SN am Körper des Patienten positioniert werden, ist es diesem weitestgehend möglich, seinem Alltag ohne große Einschränkungen nachzugehen.
Hierdurch wird die Möglichkeit zur Langzeitanalyse verschiedener Parameter eines Patienten, außerhalb eines Labors, erheblich vergrößert.


% !TEX root = MAIN.tex

\chapter{Hauptteil}
\begin{figure}[htbp]
  \centering
     \includegraphics[width=1\textwidth]{images/Init_plot_dataset1.png}
  \caption{Erstes Bild}
  \label{fig:Bild1}
\end{figure}


\include{messung}

\include{ergebnisse}

\include{diskussion}

\include{fazit}
%**********
%* Anhang *
%**********

\cleardoublepage
\appendix
\include{anhang}


%**********************************************************
% Literaturverzeichnis (hier muss nichts geändert werden) *
%**********************************************************

\bibliographystyle{alphadin}    %\bibliographystyle{} plain dinat
\bibliography{literatur}%


\end{sloppypar}
\end{document}
