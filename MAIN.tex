%***********************************************************************
%* File            :    MAIN.tex
%*
%* HAUPTDOKUMENT
%*
%* Dieses Dokument bindet alle untergeordneten Dokumente
%* ein. Es wird beim Aufruf von pdflatex als Quelle angegeben.
%*
%* Autoren         :    Daniel Hering, Tobias Wartzek
%*√
%* Datum           :    05.12.2009
%***********************************************************************



\documentclass[12pt,a4paper,twoside,openright,index=totoc,listof=totoc,DIV15,BCOR=8.25mm,headinclude,footinclude=false,headsepline,notitlepage,numbers=noenddot,bibliography=totoc,headings=normal,parskip=full]{scrbook}

%************************************************************************
% Das Package muss leider hier eingebunden werden da sonst in den
% Befehlen für hypersetup keinerlei Umlaute zur Verfügung stehen√
%************************************************************************

\usepackage{graphicx}
\usepackage{float}
\usepackage[utf8]{inputenc}         % erlaubt direkte Eingabe Deutscher
                                    % Sonderzeichen in .tex Dateien
                                    %(latin1=iso8859-1 fuer Unix Editoren
                                    % oder Windows Editoren die auch latin1
                                    % beherrschen)

%************************************************************************
% Dateiinfos eintragen (u.a. von hypersetup und f�r die Titelseite verwendet)
%************************************************************************
\usepackage{pdfpages}
% Titel der Arbeit
\newcommand*{\doclongtitle}{Bestimmung der Positionsbeziehungen von körpergetragenen Orientierungssensoren zur Beurteilung der posturalen Kontrolle}

% Art der Arbeit (Diplomarbeit, Studienarbeit, Bachelorarbeit, etc.)
\newcommand*{\docsubject}{Bachelorarbeit}

% Name des Autors
\newcommand*{\docauthor}{Julia Ribbrock}

% E-Mail Adresse des Autors
\newcommand*{\docauthoremail}{julia.ribbrock@rwth-aachen.de}

% Schlüsselworte der Arbeit
\newcommand*{\dockeywords}{MedIT Latex Studienarbeit Diplomarbeit Vorlage}

% Name des Betreuers
\newcommand*{\docsupervisor}{Markus Lüken, M.Sc.}

%\newcommand*{\docauthorfax}{+49 241 80-82442}
%\newcommand*{\docauthorphone}{+49 241 80-23211}

% Bild auf der Titelseite
\newcommand*{\doctitlepic}{images/title.jpg}

% folgende Befehle nicht ändern!!
\newcommand*{\dochomepage}{www.medit.hia.rwth-aachen.de}
\newcommand*{\docshorttitle}{\doclongtitle}

%**********************************************
%* Pakete (HEADER)                           *
%**********************************************

%************************************************************************
%* File            :    HEADER.tex
%*
%* Einbinden der erforderlichen Package-Dateien.
%* Definition des PDF-Infoblocks (im Acrobat Reader unter
%* Datei/Dokumenteigenschaften/Uebersicht)
%*
%* Autor  	       :    Daniel Hering, Tobias Wartzek
%* Datum           :    07.12.2009
%************************************************************************

\usepackage[T1]{fontenc}
\usepackage{lmodern}
\usepackage{helvet}
%\usepackage{palatino}
%\usepackage{times}


\usepackage[ngerman]{babel}              % Deutsche Silbentrennung ...
                                         % Sollte direkt nach documentclass 
                                         % geladen werden.
\usepackage{scrpage2}                    % Kopf- und Fusszeilen
\usepackage{ifpdf}                       % PDF Ausgabe oder nicht ?
                                         % Dieses Paket muss ueber dem graphicx
                                         % Paket stehen.

\usepackage{color}                       % farbig drucken                      
\usepackage{graphicx}                    % alle mgl. Arten von Bildern einbinden
\usepackage{subfigure}
\usepackage{eso-pic}										 % mehrschichtige Seitenlayouts
\usepackage{tikz}												 % Zeichenpaket und Optionen f�r Bilder.

%******************************************************************************************
% Wenn Matlab Plots erstellt werden sollen, muss das folgende package eingebunden werden. *
% Allerdings kann es vorkommen, dass das package noch nicht installiert ist, so dass *
% dies nachinstalliert werden m�sste.

%\usepackage{pgfplots}							 % Einbinden von Matlab-Plots      		*
%\pgfplotsset{compat=newest}
%\usetikzlibrary{plotmarks}
%******************************************************************************************

% Pakete f�r Tabellen
\usepackage{tabularx}                    % Breite Tabellen (seitenbreite)
\usepackage{multicol}                    % \multicols: Mehrspaltig auf einer Seite
\usepackage{hhline}                      % schoenere Doppellinien fuer Tabellen
\usepackage{longtable}                   % Lange Tabellen
\usepackage{multirow}                    % Mehrzeilige Tabellenzellen
\usepackage{colortbl}										 % Farbige Tabellenzeilen

\usepackage[gennarrow]{eurosym}          % Das Eurosymbol - benutzen mit \EUR{1,00} oder \euro{}
\usepackage{amsmath}                     % mathematische Sonderzeichen
\usepackage{amsxtra}
\usepackage{amssymb}                   	 % Zusaetzliche mathematische Symbole
\usepackage{units}											 % Verwendung: 
\usepackage{acronym}										 
\usepackage{scrhack}										 % Beseitigt Inkompatibilit�ten zwischen Komascript und Listings-Paket
\usepackage{listings}										 % Darstellung von Programmcode als Listing
\usepackage{url} 
\usepackage[pdfpagelabels]{hyperref}     % Konfiguration ueber hypersetup
                                         % hyperref sollte als eines der
                                         % letzten Pakete eingebunden werden

\usepackage{ragged2e}										 % Links-/rechtsb�ndig und trotzdem passende Zeilenumbr�che





\newcommand{\changefont}[3]{
	\fontfamily{#1} \fontseries{#2} \fontshape{#3} \selectfont}
\changefont{ptm}{m}{n}

%%%%%%%%%%%%%%%%%%%%%%%%%%%%%%%%%%%%%%%%%%%%%%%%%%%%%%%%%%%%%%%%%%%%%%%%%%%%%%%%%%%%%%%
%%%%%%%%%%%%%%%%%%%%%%%%%%%%%%%%%%%%%%%%%%%%%%%%%%%%%%%%%%%%%%%%%%%%%%%%%%%%%%%%%%%%%%%
%**************************************************************************************
%*     																																								*
%* Ab hier folgen Definitionen einiger Einstellungen die nicht 								  			*
%* vom Benutzer ge�ndert werden m�ssen. 																		  				*
%* 																																						  			*
%**************************************************************************************

\definecolor{rwthblue}{rgb}{0,0.4,0.8}    % Das RWTH-Blau
\definecolor{hellgrau}{rgb}{0.9,0.9,0.9} 	% hellgrau

%************************************************************************
%*																																			*
%* F�r die Ausgabe mit pdflatex n�tige Befehle zur sinnvollen Benutzung	*
%* des Hyperref-Pakets. Auch hier sind normal keine Anpassungen durch 	*
%* den Benutzer n�tig 																									*
%*																																			*
%************************************************************************


\ifpdf
  % \Htmlfalse
  \pdfoutput=1

  \hypersetup{
                pdftitle={\docshorttitle, RWTH Aachen},
                pdfauthor={\docauthor},
                pdfsubject={\docsubject},
                pdfkeywords={\dockeywords},
                pdfpagemode={UseNone},
                plainpages={false},
                colorlinks=true,  % die Option colorlinks schaltet um zwischen "Rahmen
                linkcolor=black,	% um link" -> false und "Text farbig" -> true 
				        citecolor=black,  % daher sind die Farbdefinitionen n�tig um Links nicht
				        filecolor=black,	% im Text hervorzuheben. (entw. weisse Rahmen oder
        				urlcolor=black		% eben schwarzer Text)
   }      
\fi


%*************************************************************
% Die folgenden zwei Zeilen werden ben�tigt sofern Matlab-	 *
% Plots eingebunden werden sollen														 *
%*************************************************************

\newlength\fheight											% ben�tigt zur Einbindung von Matlab-Plots 
\newlength\fwidth												% auskommentieren bei nichtbenutzen


%****************************************************************
% Die folgenden zwei Zeilen �ndern die Gr��e der 	        	    *
% Unterschriften um eine klare Abgrenzung zum Text zu erreichen *
%****************************************************************

\setkomafont{caption}{\small}									% Bild-/Tabellenunterschrift �ndern
\setkomafont{captionlabel}{\small\bfseries}


%*******************************************************************
%* Abb. statt Abbildung	/ Tab. statt Tabelle											 *
%*******************************************************************
\addto\captionsngerman{
\renewcommand{\figurename}{Abb.}
\renewcommand{\tablename}{Tab.}
} 


%*******************************************************************
%* Vertikaler Abstand vor Chapter �berschrift �ndern  						 *
%*******************************************************************
\renewcommand {\chapterheadstartvskip}{\vspace*{-\topskip}} 




%*************************************************************
%*************************************************************
%*																													**
%* 					Beginn des Dokuments 						  							**
%*																													**
%*************************************************************
%*************************************************************

\begin{document}
\begin{sloppypar}                       % schöner Blocksatz trotz langer Worte



%*************************************************************
%																														 *
%										 ENDE ALLER DEFINITIONEN 								 *
%																														 *
%*************************************************************


\frontmatter    												%römische Nummerierung aktivieren
%***********************************************************************
%* File            :    titel2.tex
%*
%* Titelseite
%*
%* Autor           :    Daniel Hering
%***********************************************************************

\newcommand{\defaultarraystretch}{\arraystretch}

\begin{titlepage}
\enlargethispage{31mm}
%***********************************************************************
%* Definition der Kopfzeile																						 *
%***********************************************************************
\vspace*{-3cm}
\hspace*{-1.8em}
\begin{tabularx}{19cm}{Xl@{}}
\textsf{\textbf{\LARGE{\docsubject}}} & \includegraphics[height=45pt]{images/RWTH_Aachen_University}
\\
\mbox{\vspace{-10mm}\colorbox{rwthblue}{\hspace{18.4cm}}
\rule{0pt}{10pt}}
%\vspace{-10mm}\colorbox{rwthblue}{\hspace{12.7cm}}  {\includegraphics[height=30pt]{images/medit_l_m_blau_meditheadfoot}}
\end{tabularx}

%***********************************************************************
%* Ende der Definition der Kopfzeile																	 *
%***********************************************************************
\vspace*{15mm}
\par
\begingroup
\huge
\leftskip20pt
\noindent \textsf{\LARGE\docauthor\\
\textbf{\huge \docshorttitle}}
\par
\endgroup
\vfill % F�llt die Seite vertikal auf

%***********************************************************************
%* Hier eine zur Diplomarbeit geh�rende Grafik einbinden 							 *
%***********************************************************************
% \includegraphics[width=190mm]{images/beispiel1}\vspace{10pt}

\AddToShipoutPicture*{
     \parbox[b][\paperheight]{\paperwidth}{%
       \vfill
        \centering
        \begin{tikzpicture}[overlay]%
       		  %\draw[help lines] (0,0) grid (10,20);
            \node (0,0) [opacity=0.4]{%
             \hspace{-8,25mm}\includegraphics[width=10cm, height=10cm,keepaspectratio]{\doctitlepic}};%
         \end{tikzpicture}
       \vspace{32.35em}
     }}







\renewcommand{\arraystretch}{1.3}
\hspace*{-1.8em}
\begin{tabularx}{19cm}{Xl}
\multicolumn{2}{l}{\colorbox{rwthblue}{\hspace{18.4cm}}}\\
\multicolumn{2}{l}{\textsf{\textbf{LEHRSTUHL FÜR MEDIZINISCHE INFORMATIONSTECHNIK}}}\\
\multicolumn{2}{l}{\textsf{{Fakultät für Elektrotechnik und Informationstechnik, RWTH Aachen}}}\\
\multicolumn{2}{l}{\textsf{Univ.-Prof. Dr.-Ing. Dr. med. Steffen Leonhardt}}\\
\textsf{Betreuer-/in: \docsupervisor} & \\
\textsf{Datum: \today} & \multirow{-2}*~{\includegraphics[height=30pt]{images/medit_l_m_blau_meditheadfoot}} \\
\end{tabularx} %}
\renewcommand{\arraystretch}{\defaultarraystretch}

\thispagestyle{empty}
\cleardoublepage

\end{titlepage}

\cleardoubleemptypage


%**************
%* Danksagung *
%**************

\chapter{Danksagung}
Die Erstellung dieser Arbeit wäre mir ohne die Hilfe einiger Menschen nicht möglich gewesen.
Aus diesem Grunde möchte ich mich an dieser Stelle gern bei einigen Personen bedanken.

Ein großer Dank geht an meinen Betreuer, Herrn Markus Lüken (M.Sc.), der mir über die gesamte Bearbeitungszeit, jederzeit hilfreich zur Seite stand und mir, mit aufmunternden Worten, immer wieder zum nötigen Optimismus zur Fertigstellung dieser Arbeit verholfen hat.

Desweiteren möchte ich mich bei Herrn Prof. Dr.-Ing. Dr. Med. Leonhardt, dem Leiter des Lehrstuhls für medizinische Informationstechnik, für die Möglichkeit bedanken diese Arbeit am Institut verfassen zu können.

Meiner Familie möchte ich dafür danken, dass sie mich die letzten Jahre immer darin bekräftigt haben meinen Weg in diesem Studium zu gehen und nie aufzugeben.

\cleardoubleemptypage

%************************************************
%* Erklärung (hier muss nichts geändert werden) *
%************************************************

%\chapter*{Erkl�rung}
%\thispagestyle{empty}
%\hypertarget{hypsec:erklaerung_der_selbstst}{}%
%
%Ich versichere hiermit, dass ich die vorliegende Arbeit selbstst�ndig
%und ohne Benutzung anderer als der angegebenen Hilfsmittel angefertigt
%habe. Alle Stellen, die w�rtlich oder sinngem�� aus
%ver�ffentlichten und nicht ver�ffentlichten Schriften entnommen
%sind, wurden als solche kenntlich gemacht.\\[3cm]
%
%\begin{tabularx}{\textwidth}{lXl}
%  \rule{5cm}{0.4pt} & & \rule{5cm}{0.4pt}\\
%  Ort, Datum & & Unterschrift
%\end{tabularx}
%\includepdf[angle=90,landscape]{Testbild.png}
\includepdf{images/Formular_Eidesstattliche_Versicherung_neu}

%\begin{figure}[!ht]
%  \centering
%   \includegraphics[scale=1]{images/Formular_Eidesstattliche_Versicherung_neu}
%   % Die folgende Anweisungen bindet die Grafik nocheinmal zus�tzlich
%   % als Dateianhang ein. Das hat den Vorteil, dass sie einfach aus dem pdf zur
%   % wieterverwendung extrahierbar ist und der Autor (author) mit angegeben
%   % werden kann. Leider ist die Grafik dadurch auch gleich zweimal im pdf
%   % enthalten und verbraucht dadurch doppelt Platz.
%   % \textattachfile[mimetype=application/pdf,print=false,author=\docauthor,description=\figurename~\ref{fig:beispiel1}]{beispiel1.pdf}{}
%  %% Reihenfolge der Befehle wichtig: zuerst die \caption, danach das \label
%  \label{fig:EidesstattlicheVersicherung}
%\end{figure}

\cleardoubleemptypage

%************
%* Abstract *
%************

\chapter{Zusammenfassung}

Lorem ipsum dolor sit amet, consetetur sadipscing elitr, sed diam nonumy eirmod tempor invidunt ut labore et dolore magna aliquyam erat, sed diam voluptua. At vero eos et accusam et justo duo dolores et ea rebum. Stet clita kasd gubergren, no sea takimata sanctus est Lorem ipsum dolor sit amet. Lorem ipsum dolor sit amet, consetetur sadipscing elitr, sed diam nonumy eirmod tempor invidunt ut labore et dolore magna aliquyam erat, sed diam voluptua. At vero eos et accusam et justo duo dolores et ea rebum. Stet clita kasd gubergren, no sea takimata sanctus est Lorem ipsum dolor sit amet. Lorem ipsum dolor sit amet, consetetur sadipscing elitr, sed diam nonumy eirmod tempor invidunt ut labore et dolore magna aliquyam erat, sed diam voluptua. At vero eos et accusam et justo duo dolores et ea rebum. Stet clita kasd gubergren, no sea takimata sanctus est Lorem ipsum dolor sit amet. 

Duis autem vel eum iriure dolor in hendrerit in vulputate velit esse molestie consequat, vel illum dolore eu feugiat nulla facilisis at vero eros et accumsan et iusto odio dignissim qui blandit praesent luptatum zzril delenit augue duis dolore te feugait nulla facilisi. Lorem ipsum dolor sit amet, consectetuer adipiscing elit, sed diam nonummy nibh euismod tincidunt ut laoreet dolore magna aliquam erat volutpat. 

Ut wisi enim ad minim veniam, quis nostrud exerci tation ullamcorper suscipit lobortis nisl ut aliquip ex ea commodo consequat. Duis autem vel eum iriure dolor in hendrerit in vulputate velit esse molestie consequat, vel illum dolore eu feugiat nulla facilisis at vero eros et accumsan et iusto odio dignissim qui blandit praesent luptatum zzril delenit augue duis dolore te feugait nulla facilisi. 


\cleardoubleemptypage
\thispagestyle{empty}
\cleardoubleemptypage

%**********************
%* Inhaltsverzeichnis *
%**********************


\addcontentsline{toc}{chapter}{Inhaltsverzeichnis} % Eintrag im Inhaltsverzeichnis erstellen
\tableofcontents                        % Inhaltsverzeichnis anlegen
\cleardoubleemptypage



%*********************
%* Symbolverzeichnis *
%********************

\chapter*{Symbolverzeichnis}										% da *-Variante m�ssen Kopfzeilen und TOC-Eintrag von Hand generiert werden
\markboth{Symbolverzeichnis}{Symbolverzeichnis} 				% Kopfzeile manuell anpassen
\addcontentsline{toc}{chapter}{Symbolverzeichnis}				% TOC-Eintrag



\section*{Abkürzungen}

%* Benutzung der Acronym-Umgebung
%* \acrodef{Name zum aufrufen im Dokument}[gew�nschte Abk�rzung]{ausgeschriebener Begriff}
%* \ac{Name} zur Verwendung im Dokument - beim 1.maligen Verwenden wird der ausgeschriebene
%* Begriff mit der Abk�rzung dahinter in Klammern gesetzt, danach nur noch die Abk�rzung

\acrodef{BSN}[BSN]{Body Sensor Network}
\acrodef{EKF}[EKF]{Extended Kalman Filter}
\acrodef{IMU}[IMU]{Inertial Measurement Unit}
\acrodef{MedIT}[MedIT]{Institut f{\"u}r medizinische Informationstechnik}
\acrodef{MEMS}[MEMS]{Microelectronic Mechanical Systems}
\acrodef{MN}[MN]{Masterknoten}
\acrodef{SN}[SN]{Sensorknoten}
\acrodef{ToF}[ToF]{Time of Flight}
\acrodef{UKF}[UKF]{Unscented Kalman Filter}
\acrodef{UT}[UT]{Unscented Transformation}
\acrodef{VBA}[VBA]{Vibrating Beam Accelerometer}


%* \acs{Name} ruft explizit die Abkürzung auf.
%* \acl{Name} ruft explizit den ausgeschriebenen Begriff auf.

\begin{tabularx}{\textwidth}{p{.18\textwidth}X}
\acs{BSN} & \acl{BSN} \\
\acs{EKF} & \acl{EKF} \\
\acs{IMU} &\acl{IMU} \\
\acs{MedIT} & \acl{MedIT} \\
\acs{MEMS} & \acl{MEMS} \\
\acs{MN} & \acl{MN} \\
\acs{SN} & \acl{SN} \\
\acs{ToF} & \acl{ToF} \\
\acs{UKF} & \acl{UKF} \\
\acs{UT} & \acl{UT} \\
\acs{VBA} & \acl{VBA} \\
\end{tabularx}

\cleardoubleemptypage

\mainmatter							% Arabische Nummerierung, Beginn des Hauptteils

%**************
%* Einleitung *
%**************

\chapter{Einleitung}
\section{Aufbau der Arbeit}
Die vorliegenden Arbeit befasst sich mit der Bestimmung der relativen Positionen der Sensorknoten innerhalb eines Body Sensor Networks.
Hierdurch soll eine Bewertung der posturalen Kontrolle des Trägers des Systems ermöglicht werden.

Das Kapitel 2 dient hierbei dazu, sowohl medizinische, mathematische, als auch technische Grundlagen zu erläutern.
Auf diie genauere Thematik und den verwendeten Ansatz dieser Arbeit wird in Kapitel 3 eingegangen.
Kapitel 4 befasst sich mit dem Messaufbau zur Validierung des Algorithmus mithilfe eines optischen Trackingsystems.
Die mithilfe der Messung erzielten Ergebnisse werden in Kapitel 5 näher diskutiert, um in Kapitel 6 schließlich zu einem Fazit über die verwendete Algorithmik und Sensorik zu gelangen.


%*************
%* Hauptteil *
%*************
\chapter{Grundlagen}
\section{Medizinische Grundlagen}
\subsection{Posturale Kontrolle}
Der Begriff der \textit{posturalen Kontrolle} bezeichnet die Fähigkeit eines Menschen, auch trotz kurzfristiger Störungen, das motorische Gleichgewicht herstellen, aufrechterhalten und wiedererlangen zu können \cite{PK1}.
Da zur Realisierung der posturalen Kontrolle ein ständiger Abgleich zwischen dem sensorischen Prozess der Erfassung der Körperlage und eventueller Bewegungen und der motorischen Reaktion auf eben diese erfolgen muss, spricht man häufig auch von der sensomotirischen Gleichgewichtsregulation \cite{PK2}.

\section{Mathematische Methoden}
Innerhalb der vorliegenden Arbeit werden, unter Anderem zur Datenverarbeitung, verschiedene Berechnungen durchgeführt und mathematische Annahmen getroffen, welche auf grundlegenden mathematischen Ansätzen beruhen.
So wird zur Datenverarbeitung, unter anderem vom Kalman Filter (KF) und einer seiner Weitereinwicklungen, dem Unscented Kalman Filter (UKF), Gebrauch gemacht.

\subsection{Kalman Filter}
Betrachtet man dynamische Systeme, muss stets davon ausgegangen werden, dass diese keine ideale Beschreibung des Systemverhaltens zulassen.
Mitunter werden die Systemgleichungen bei der mathematischen Betrachtung vereinfacht, was in einer, als Systemrauschen bezeichneten, additiven, nicht beeinflussbaren Störung des Systems resultiert.
Durch die Anbringung von Sensoren ist eine weitere Fehlerquelle gegeben, deren Einfluss über die Ausgangsgröße des Systems berücksichtigt wird. Störungen dieser Art werden Messrauschen genannt.
Sowohl das Systemrauschen, als auch das Messrauschen werden in Form von Zufallsprozessen in die Berechnungen miteinbezogen. Dies hat jedoch zur Folge, dass auch die Eingangs- und Ausgangsgrößen des Systems als stochastische Größen betrachtet werden müssen.
Um dennoch eine qualitativ hochwertige Aussage über die Zustände des Systems treffen zu können, muss ein, für den gestörten Eingangsvektor, optimaler Schätzwert gefunden werden. Als optimal wird hierbei, aufgrund der wahrscheinlichkeitstheoretischen Betrachtung, ein Wert minimaler Varianz angestrebt. \cite{Sys2}
\\Einen Ansatz zu diesem Problem lieferte R. Kalman im Jahr 1960 mit dem, nach ihm benannten, Kalman Filter \cite{kalman}.

\subsubsection{Unscented Kalman Filter}


\section{Bewegungserfassung}
Bei der Bewegungserfassung, welche häufig auch als \textit{Motion Capturing} bezeichnet wird, erfolgt eine Aufzeichnung und Umwandlung realer Bewegungen zu einer digitalen, für Computer erfassbare, Darstellung eben dieser.
Dies wird durch das Verfolgen markanter Punkte, auch \textit{tracken} genannt, während der Bewegung eines Körper und anschließender Erstellung eines 3D-Modells zur weiteren Verarbeitung.
Motion Capturing kann unter Verwendung verschiedener Systeme realisiert werden. Die größte Verwendung von Motion Capturing Systemen liegt in Bereichen wie der Medizin oder auch dem Entertainment, z.B. in Form der Animation von Filmcharakteren. \cite{mocap_def}

\subsection{Definitionen}
Zum besseren Verständnis der Thematik des Motion Capturing werden an dieser Stelle einige grundlegende Begriffe erläutert.
\subsubsection{Aktionsbereich}
Hierunter wird der Bereich verstanden, in welchem der Tracker, unter Einhaltung der angegebenen Auflösung und Genauigkeit, zuverlässige Ergebnisse liefern kann \cite{P25}.
Bei optischen Systemen wird hiermit der Raum beschrieben, in welchem die Bildausschnitte der Kameras eine Bestimmung der Position des Markers zulassen \cite{mocap}.
\subsubsection{Auflösung}
Unter der Auflösung eines Trackingsystems versteht man die kleinste Änderung in Position und Orientierung, welche durch das System aufgezeichnet werden kann \cite{P25}.
\subsubsection{Genauigkeit}
Bei der Genauigkeit handelt es sich um den Messfehler, welcher während der Messung von Orientierung und Position auftritt \cite{P25}.
\subsubsection{Latenzzeit}
Die Latenzzeit beschreibt die Verzögerung zwischen der tatsächlichen Änderung in Position und Orientierung und dem Zeitpunkt zu welchem das System die Änderung erkennt \cite{vr}.
\subsubsection{Marker}
Bei Markern handelt es sich um Hilfsmittel zur Kennzeichnung markanter Punkte am zu trackenden Objekt. Hierbei wird zwischen aktiven und passiven Markern unterschieden.
Unter aktiven Markern versteht man solche, welche, meist in Form vom LEDs, infrarotes Licht emittieren. Dieses wird von speziellen, mit den Markern synchronsierten, Kameras erfasst. \cite{ART}
\\Aktive Marker eignen sich besonders für den Einsatz in Verbindung mit relexionsreichen Materialien und bei staubigen Umgebungsverhältnissen \cite{qualsys_act}.
\\Im Gegensatz zu den aktiven Markern sind passive Marker retroreflektiv. Anstatt wie aktive Marker infrarotes Licht zu emittieren, reflektieren sie einfallende Infrarotstrahlung in die Einfallsrichtung gebündelt. \cite{ART}
\subsubsection{Time of Flight}
Unter der Time of Flight (ToF) versteht man die Zeit, welche ein Signal von seiner Quelle bis zum messenden Sensor benötigt. Mithilfe der ToF kann aufgrund der Proportionalität der Dauer zur Strecke eine Aussage über die zurückgelegte Strecke getroffen werden. \cite{P24}
\subsubsection{Updaterate}
Die Updaterate eines Systems beschreibt die Anzahl der Abtastungen eines Sensors pro Sekunde. Je höher die Updaterate ist, desto besser kann das dynamische Systemverhalten beschrieben werden. \cite{vr}
\subsubsection{Verdeckung}
Verdeckung beschreibt den Zustand bei welchem ein Marker von einer oder mehreren Kameras nicht mehr wahrgenommen werden kann, weil Teile des zu trackenden Objekts oder andere Gegenstände die Sicht auf den Marker verhindern \cite{occlusion}.

\subsection{Arten der Aufnahmesysteme}
Im Bereich der Bewegungserfassung wird zwischen drei verschiedenen Gruppen der Aufnahmesysteme unterschieden. Hierbei erfolgt die Zuordnung über die Anbringungsposition der Sensoren und Datenquellen. \cite{P25}
\subsubsection{Outside-In Systeme}
Das Prinzip der Outside-In Systeme sieht eine Anbringung der Sensoren an festen Positionen im Raum vor. Die Sensoren werden hierbei auf das zu trackende Objekt ausgerichtet. Dieses ist wiederum mit Markern versehen, über welche die Datenerfassung erfolgen kann.
Hierbei ist auf eine ausreichend hohe Anzahl von Sensoren zu achten, um eine möglichst fehlerfreie Auswertung der Aufzeichnungen garantieren zu können. \cite{P25}
\\Dieses Verfahren findet Anwendung in optischen Trackingsystem, welche darauf beruhen, dass Bewegungen des, mit Markern versehenen Objektes, über Kameras erfasst werden \cite{optsys2}.
\subsubsection{Inside-In Systeme}
Bei Inside-In Systemen, werden nicht nur die Datenquellen am Aktuer platziert, sondern auch die Sensoren.
Ein klassisches Beispiel hierfür stellen Exo-Skelett Systeme dar, bei welchen die relativen Bewegungen des Darstellers über Potentiometer bestimmt werden können.
Ein wesentlicher Vorteil dieser Systeme, gegenüber den Outside-In Systemen, besteht darin, dass das Auftreten von Verdeckungsproblemen vermieden wird und somit das Tracken mehrerer Akteure zum gleichen Zeitpunkt, ohne die Gefahr eines Datenverlustes möglich ist.
Oft wird hierbei jedoch die Bewegungsfreiheit der Darsteller durch das mechanische Skelett stark eingeschränkt. \cite{optsys2}
\subsubsection{Inside-Out Systeme}
Inside-Out Systeme nutzen am Darsteller positionierte Sensoren, welche Signale von fest in der Umgebung platzierten Datenquellen empfangen und auswerten können \cite{P25}.
Ein Beispiel für Systeme dieser Art sind magnetische Trackingsysteme. Hierbei wird durch externe Quellen ein Magnetfeld erzeugt, dessen Feldstärke dann an den jeweiligen Sensoren gemessen wird, um eine Entfernung zur Quelle zu bestimmen. Über die relativen Positionen der Sensoren zu verschiedenen Quellen ist anschließend eine räumliche Positionsbestimmung möglich. \cite{optsys2}

\subsection{Optische Trackingsysteme}
Unter optischen Trackingsystemen werden all die Verfahren des Motion Capturing zusammengefasst, welche auf einer Bewegungserfassung unter Verwendung von Lichtquellen und optischen Sensoren basieren. \cite{P31}
Im Regelfall werden für ein präzises optisches Tracken, je nach verwendetem System, vier bis zwölf Spezialkameras benötigt.\cite{optsys1}
\\Das Themengebiet der optischen Trackingsysteme lässt sich in das markerbasierte und das markerlose Tracking unterteilen.
\\Beim markerbasierten Tracking wird zwischen der Verwendung von aktiven und passiven Markern unterschieden \cite{P25}. Die Funktionsweise beider Teilbereiche beruht jedoch darauf, dass Markerpositionen mithilfe von Spezialkameras erfasst werden. Hierbei werden die Kameras meist am oberen Rand der Aufnahmeumgebung, in regelmäßigen Abständen zueinander, angebracht.
Eine typischerweise beim passiven Tracking verwendete Kamera ist so aufgebaut, dass ringförmig um das Objektiv herum, eine Vielzahl von Infrarot-Leuchtdioden angebracht ist, über welche die passiven Marker angestrahlt und die Reflexion anschließend vom Objektiv erfasst werden kann. \cite{optsys2}
Ein Beispiel hierfür, stellen die in dieser Arbeit, im Referenzsystem verwendeten Oqus Kameras der Firma Qualisys dar \cite{oqus}.
\\Je nach Art und Aufbau des verwendeten Systems kann es zu großen Schwankungen bezüglich des Aktionsbereichs zwischen den einzelnen Systeme kommen \cite{P24}. Bei geeigneter Wahl des Systems ist das erzielen sehr großer Aktionsbereiche jedoch durchaus möglich \cite{P25}.
Ein weiterer positiver Aspekt des markerbasierten Trackings liegt in der leichten Erweiterbarkeit des Messaufbaus. So können ohne großen Aufwand zusätzliche Marker am Akteur angebracht werden, ohne dass dieser wesentlich in seiner Bewegungsfreiheit beeinträchtig wird. \cite{optsys2}
\\Das markerbasierte Tracking bietet jedoch nicht nur Vorteile. Ein wesentlicher Nachteil Systeme dieser Art ist durch die Verdeckung gegeben. So kann es vorkommen das, entweder vom Darsteller selbst, durch weitere Akteure oder aber auch durch Gegenstände innerhalb des Aktionsbereichs, ein Marker verdeckt wird und so nicht von ausreichend Kameras eingesehen werden kann. Ein Datenverlust aufgrund eines aus dem Bildbereich einer Kamera entwichenen Markers kann teilweise durch Interpolationsverfahren bei der späteren Datenverarbeitung behoben werden. \cite{optsys2} Um die Problematik der Verdeckung zu reduzieren, bevor es zu Datenverlusten kommen kann, ist es mitunter sinnvoll weitere Kameras in die Aufnahmeumgebung zu integrieren \cite{P25}.
\\Beim markerlosen Tracking erfolgt die Bewegungserfassung, abgesehen von Kameras, ohne weitere Hilfsmittel. Ein Beispiel hierfür sind ToF-Kameras. Diese bestimmen die Position es zu trackenden Objektes über eine Lichtzeitmessungen. Hierbei wird die Zeit bestimmt, welche ein, von der Kamera ausgestrahlter, Lichtimpuls benötigt um nach Reflexion wieder die Kamera zu erreichen. \cite{basler}
Ein weiteres markerloses Trackingverfahren ist das sogenannte Laser-Ranging. Hierbei wird ein Beugungsgitter zwischen Objekt und Laser platziert und durch den Laser angestrahlt, wodurch ein Intereferenzmuster auf dem Objekt erzeugt wird. Dieses Interferenzmuster wird anschließend mithilfe von Kameras gescannt, welche die Intensität des Musters bestimmen. Hieraus wird dann der Contrast Ratio bestimmt, welcher dazu dient die Distanz zwischen getracktem Objekt und Beugungsgitter zu bestimmen. Ein Nachteil dieses Verfahrens ist der stark eingeschränkte Aktionsbereich, welcher daher rührt, dass mit steigender Entfernung zwischen Beugungsgitter und Objekt die Messgenauigkeit stark abnimmt und somit keine qualitativ wertvolle Aussage über die tatsächliche Distanz mehr getroffen werden kann. \cite{P25}
\\Unabhängig davon, ob es sich um markerbasiertes oder markerloses Tracking handelt zeichnen sich optische Verfahren zur Bewegungserfassung besonders, durch das Erzielen sehr hoher Genauigkeit und Auflösung innerhalb ihrer jeweiligen Aktiobsbereiche aus \cite{P24}. Auch die Verfügbarkeit hoher Updateraten stellt in beiden Fällen einen positiven Aspekt der Systeme dar \cite{P25}.
Der häufig hohe Rechenaufwand, welcher durch eventuelle Nachbearbeitung der Daten entsteht ist ein wesentlicher Nachteil dieser Art der Bewegungserfassung \cite{optsys1}. Da es bei optischen Verfahren zu, durch Umgebungslicht verursachten, Störungen kommen kann ist ein erfolgreicher Einsatz dieser Techniken im wesentlichen nur in Messlaboren unter konstanten Umgebungsverhätnissen möglich \cite{P25}.

\subsection{Tracking mit Intertialsensorik}
In Trackingsystemen mit Inertialsensorik kommen sowohl Beschleunigungssensoren, auch Accelerometer genannt, als auch Gyroskope zum Einsatz. Dabei dienen die Beschleunigungssensoren der Bestimmung der Position des Objektes, während mithilfe der Gyroskope eine Aussage über die Objektorientierung getroffen werden kann. \cite{P24}
Hierbei beruhen alle Bestimmungen auf dem Messen von Trägheitskräften \cite{P32} unter der Anwendung des 2. Gesetzes von Newton, in der Form
\begin{equation}\label{Newton2.1}
    F = ma
\end{equation} oder
\begin{equation}\label{Newton2.2.}
   Q = I\alpha
\end{equation} \cite{P24}.
\\Da sowohl die Messungen der Beschleunigungssensoren, als auch die der Gyroskope jeweils nur eine Achse einschließen, müssen für die Ausführung dreidimensionaler Messungen jeweils drei Accelerometer und drei Gyroskope zu einer Inertial measurement unit (IMU) kombiniert werden \cite{P32}. Hierfür werden die Sensoren einer Art jeweils orthogonal zueinander auf einer Platform aufgeordnet \cite{P24}.
\\Die am häufigsten genutzte Art der Accelerometer sind die Pendelbeschleunigungsmesser. Bei diesen ist eine bekannte Masse an einer Seite einer gedämpften Feder befestigt, während die andere Seite am Gehäuse des Sensors angebracht ist. \cite{P32}
Während keine Beschleunigung auf den Sensor wirkt, ruhen Feder und Masse. Das Sensorgehäuse und die Referenzmasse sind somit ohne Versatz zueinander ausgerichtet. Sobald jedoch eine Kraft auf das Gehäuse einwirkt und dieses einer Beschleunigung ausgesetzt wird entsteht, aufgrund der Trägheit der Masse, ein Versatz.
Dieser Versatz und die Stauchung bzw. Streckung der Feder können als zur Beschleunigung proportional angenommen werden. Mithilfe eines Wandler, welcher entweder potentiometrisch oder piezoelektrisch sein kann, wird aus dem Versatz ein Signal abgeleitet. Handelt es sich bei dem Wandler um ein Potentiometer, so ist der Versatz der Masse mit dem Regler des Potentiometers gekoppelt. Wird jedoch ein piezoelektrischer Wandler genutzt, erzeugt der verwendete piezokristall eine elektrische Ladung, wenn, durch den Versatz, eine Kraft auf ihn ausgeübt wird.
Da es nicht möglich ist die Beschleunigung direkt zu messen, wird unter Beachtung von Gl. \ref{Newton2.1}, die auf die Masse ausgeübte, zur Beschleunigung proportionale Kraft gemessen. Über die Relation
\begin{equation}\label{beschl}
  a = \frac{d^2r}{dt^2}
\end{equation}
kann die Position durch zweifache Integration von \textit{a} bestimmt werden, zu
\begin{equation}\label{pos}
  r = \iint{a dt^2}.
\end{equation}
Die Einwirkung der Gravitation auf die Masse muss vor der Berechnung eliminiert werden.\cite{P24}

\section{Body Sensor Network}
Eine weitere Art der Datenerhebung für die Positionsbestimmung ist durch sogenannte \textit{Body Sensor Networks} (BSN) gegeben.
Bei diesen handelt es sich um Sensornetzwerke, welche meist aus mehreren Sensorknoten (SN) und einem Masterknoten (MN) bestehen.
Innerhalb dieser SN wird überwiegend auf das Tracking mit Inertialsensorik zurückgegriffen.
Kommunikation innerhalb des BSN ist sowohl zwischen den SN untereinander, als auch zwischen einzelnen SN und dem MN möglich.
Dadurch, dass die SN am Körper des Patienten positioniert werden, ist es diesem weitestgehend möglich, seinem Alltag ohne große Einschränkungen nachzugehen.
Hierdurch wird die Möglichkeit zur Langzeitanalyse verschiedener Parameter eines Patienten, außerhalb eines Labors, erheblich vergrößert.


% !TEX root = MAIN.tex

\chapter{Hauptteil}
\begin{figure}[htbp]
  \centering
     \includegraphics[width=1\textwidth]{images/Init_plot_dataset1.png}
  \caption{Erstes Bild}
  \label{fig:Bild1}
\end{figure}


\chapter{Messung}


\chapter{Ergebnisse}


\chapter{Diskussion}


\chapter{Fazit}

%**********
%* Anhang *
%**********

\cleardoublepage
\appendix
\chapter{Anhang}



%**********************************************************
% Literaturverzeichnis (hier muss nichts geändert werden) *
%**********************************************************

\bibliographystyle{alphadin}    %\bibliographystyle{} plain dinat
\bibliography{literatur}%


\end{sloppypar}
\end{document}
