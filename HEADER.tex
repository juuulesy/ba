%************************************************************************
%* File            :    HEADER.tex
%*
%* Einbinden der erforderlichen Package-Dateien.
%* Definition des PDF-Infoblocks (im Acrobat Reader unter
%* Datei/Dokumenteigenschaften/Uebersicht)
%*
%* Autor  	       :    Daniel Hering, Tobias Wartzek
%* Datum           :    07.12.2009
%************************************************************************

\usepackage[T1]{fontenc}
\usepackage{lmodern}
\usepackage{helvet}
%\usepackage{palatino}
%\usepackage{times}


\usepackage[ngerman]{babel}              % Deutsche Silbentrennung ...
                                         % Sollte direkt nach documentclass 
                                         % geladen werden.
\usepackage{scrpage2}                    % Kopf- und Fusszeilen
\usepackage{ifpdf}                       % PDF Ausgabe oder nicht ?
                                         % Dieses Paket muss ueber dem graphicx
                                         % Paket stehen.

\usepackage{color}                       % farbig drucken                      
\usepackage{graphicx}                    % alle mgl. Arten von Bildern einbinden
\usepackage{subfigure}
\usepackage{eso-pic}										 % mehrschichtige Seitenlayouts
\usepackage{tikz}												 % Zeichenpaket und Optionen f�r Bilder.

%******************************************************************************************
% Wenn Matlab Plots erstellt werden sollen, muss das folgende package eingebunden werden. *
% Allerdings kann es vorkommen, dass das package noch nicht installiert ist, so dass *
% dies nachinstalliert werden m�sste.

%\usepackage{pgfplots}							 % Einbinden von Matlab-Plots      		*
%\pgfplotsset{compat=newest}
%\usetikzlibrary{plotmarks}
%******************************************************************************************

% Pakete f�r Tabellen
\usepackage{tabularx}                    % Breite Tabellen (seitenbreite)
\usepackage{multicol}                    % \multicols: Mehrspaltig auf einer Seite
\usepackage{hhline}                      % schoenere Doppellinien fuer Tabellen
\usepackage{longtable}                   % Lange Tabellen
\usepackage{multirow}                    % Mehrzeilige Tabellenzellen
\usepackage{colortbl}										 % Farbige Tabellenzeilen

\usepackage[gennarrow]{eurosym}          % Das Eurosymbol - benutzen mit \EUR{1,00} oder \euro{}
\usepackage{amsmath}                     % mathematische Sonderzeichen
\usepackage{amsxtra}
\usepackage{amssymb}                   	 % Zusaetzliche mathematische Symbole
\usepackage{units}											 % Verwendung: 
\usepackage{acronym}										 
\usepackage{scrhack}										 % Beseitigt Inkompatibilit�ten zwischen Komascript und Listings-Paket
\usepackage{listings}										 % Darstellung von Programmcode als Listing
\usepackage{url} 
\usepackage[pdfpagelabels]{hyperref}     % Konfiguration ueber hypersetup
                                         % hyperref sollte als eines der
                                         % letzten Pakete eingebunden werden

\usepackage{ragged2e}										 % Links-/rechtsb�ndig und trotzdem passende Zeilenumbr�che





\newcommand{\changefont}[3]{
	\fontfamily{#1} \fontseries{#2} \fontshape{#3} \selectfont}
\changefont{ptm}{m}{n}

%%%%%%%%%%%%%%%%%%%%%%%%%%%%%%%%%%%%%%%%%%%%%%%%%%%%%%%%%%%%%%%%%%%%%%%%%%%%%%%%%%%%%%%
%%%%%%%%%%%%%%%%%%%%%%%%%%%%%%%%%%%%%%%%%%%%%%%%%%%%%%%%%%%%%%%%%%%%%%%%%%%%%%%%%%%%%%%
%**************************************************************************************
%*     																																								*
%* Ab hier folgen Definitionen einiger Einstellungen die nicht 								  			*
%* vom Benutzer ge�ndert werden m�ssen. 																		  				*
%* 																																						  			*
%**************************************************************************************

\definecolor{rwthblue}{rgb}{0,0.4,0.8}    % Das RWTH-Blau
\definecolor{hellgrau}{rgb}{0.9,0.9,0.9} 	% hellgrau

%************************************************************************
%*																																			*
%* F�r die Ausgabe mit pdflatex n�tige Befehle zur sinnvollen Benutzung	*
%* des Hyperref-Pakets. Auch hier sind normal keine Anpassungen durch 	*
%* den Benutzer n�tig 																									*
%*																																			*
%************************************************************************


\ifpdf
  % \Htmlfalse
  \pdfoutput=1

  \hypersetup{
                pdftitle={\docshorttitle, RWTH Aachen},
                pdfauthor={\docauthor},
                pdfsubject={\docsubject},
                pdfkeywords={\dockeywords},
                pdfpagemode={UseNone},
                plainpages={false},
                colorlinks=true,  % die Option colorlinks schaltet um zwischen "Rahmen
                linkcolor=black,	% um link" -> false und "Text farbig" -> true 
				        citecolor=black,  % daher sind die Farbdefinitionen n�tig um Links nicht
				        filecolor=black,	% im Text hervorzuheben. (entw. weisse Rahmen oder
        				urlcolor=black		% eben schwarzer Text)
   }      
\fi


%*************************************************************
% Die folgenden zwei Zeilen werden ben�tigt sofern Matlab-	 *
% Plots eingebunden werden sollen														 *
%*************************************************************

\newlength\fheight											% ben�tigt zur Einbindung von Matlab-Plots 
\newlength\fwidth												% auskommentieren bei nichtbenutzen


%****************************************************************
% Die folgenden zwei Zeilen �ndern die Gr��e der 	        	    *
% Unterschriften um eine klare Abgrenzung zum Text zu erreichen *
%****************************************************************

\setkomafont{caption}{\small}									% Bild-/Tabellenunterschrift �ndern
\setkomafont{captionlabel}{\small\bfseries}


%*******************************************************************
%* Abb. statt Abbildung	/ Tab. statt Tabelle											 *
%*******************************************************************
\addto\captionsngerman{
\renewcommand{\figurename}{Abb.}
\renewcommand{\tablename}{Tab.}
} 


%*******************************************************************
%* Vertikaler Abstand vor Chapter �berschrift �ndern  						 *
%*******************************************************************
\renewcommand {\chapterheadstartvskip}{\vspace*{-\topskip}} 

