\chapter*{Symbolverzeichnis}										% da *-Variante m�ssen Kopfzeilen und TOC-Eintrag von Hand generiert werden
\markboth{Symbolverzeichnis}{Symbolverzeichnis} 				% Kopfzeile manuell anpassen
\addcontentsline{toc}{chapter}{Symbolverzeichnis}				% TOC-Eintrag



\section*{Abkürzungen}

%* Benutzung der Acronym-Umgebung
%* \acrodef{Name zum aufrufen im Dokument}[gew�nschte Abk�rzung]{ausgeschriebener Begriff}
%* \ac{Name} zur Verwendung im Dokument - beim 1.maligen Verwenden wird der ausgeschriebene
%* Begriff mit der Abk�rzung dahinter in Klammern gesetzt, danach nur noch die Abk�rzung

\acrodef{BSN}[BSN]{Body Sensor Network}
\acrodef{EKF}[EKF]{Extended Kalman Filter}
\acrodef{IMU}[IMU]{Inertial Measurement Unit}
\acrodef{MedIT}[MedIT]{Institut f{\"u}r medizinische Informationstechnik}
\acrodef{MEMS}[MEMS]{Microelectronic Mechanical Systems}
\acrodef{MN}[MN]{Masterknoten}
\acrodef{SN}[SN]{Sensorknoten}
\acrodef{ToF}[ToF]{Time of Flight}
\acrodef{UKF}[UKF]{Unscented Kalman Filter}
\acrodef{UT}[UT]{Unscented Transformation}
\acrodef{VBA}[VBA]{Vibrating Beam Accelerometer}


%* \acs{Name} ruft explizit die Abkürzung auf.
%* \acl{Name} ruft explizit den ausgeschriebenen Begriff auf.

\begin{tabularx}{\textwidth}{p{.18\textwidth}X}
\acs{BSN} & \acl{BSN} \\
\acs{EKF} & \acl{EKF} \\
\acs{IMU} &\acl{IMU} \\
\acs{MedIT} & \acl{MedIT} \\
\acs{MEMS} & \acl{MEMS} \\
\acs{MN} & \acl{MN} \\
\acs{SN} & \acl{SN} \\
\acs{ToF} & \acl{ToF} \\
\acs{UKF} & \acl{UKF} \\
\acs{UT} & \acl{UT} \\
\acs{VBA} & \acl{VBA} \\
\end{tabularx}
