\chapter{Einleitung}
\section{Aufbau der Arbeit}
Die vorliegenden Arbeit befasst sich mit der Bestimmung der relativen Positionen der Sensorknoten innerhalb eines Body Sensor Networks.
Hierdurch soll eine Bewertung der posturalen Kontrolle des Trägers des Systems ermöglicht werden.

Das Kapitel 2 dient hierbei dazu, sowohl medizinische, mathematische, als auch technische Grundlagen zu erläutern.
Auf diie genauere Thematik und den verwendeten Ansatz dieser Arbeit wird in Kapitel 3 eingegangen.
Kapitel 4 befasst sich mit dem Messaufbau zur Validierung des Algorithmus mithilfe eines optischen Trackingsystems.
Die mithilfe der Messung erzielten Ergebnisse werden in Kapitel 5 näher diskutiert, um in Kapitel 6 schließlich zu einem Fazit über die verwendete Algorithmik und Sensorik zu gelangen.
