\chapter{Grundlagen}

\section{Medizinische Grundlagen}

\subsection{Posturale Kontrolle}
Der Begriff der \textit{posturalen Kontrolle} bezeichnet die Fähigkeit eines Menschen, auch trotz kurzfristiger Störungen, das motorische Gleichgewicht herstellen, aufrechterhalten und wiedererlangen zu können \cite{PK1}.
Da zur Realisierung der posturalen Kontrolle ein ständiger Abgleich zwischen dem sensorischen Prozess der Erfassung der Körperlage und eventueller Bewegungen und der motorischen Reaktion auf eben diese erfolgen muss, spricht man häufig auch von der sensomotirischen Gleichgewichtsregulation \cite{PK2}.

\subsubsection{Kriterien zur Bewertung der posturalen Kontrolle}


\section{Technische Grundlagen}
Zur Bestimmung von Positionsbeziehungen, ist zunächst ein Erlangen von Daten nötig.
Hierzu kann auf verschiedene Systeme zurückgegriffen werden.
Optische Trackingsysteme sind eine Möglichkeit diese Aufgabe zu erfüllen.
Eine Alternative zu ihnen ist durch Body Sensor Networks (BSN) gegeben.

\subsection{Optische Trackingsysteme}
 nur laborgebundene Aufnahmen möglich.
 keine Langzeitanalyse.
 genauer, aber dadurch nicht hilfreich...

\subsection{Body Sensor Network}
Die Problematik der laborgebundenen Analyse auf Seiten der optischen Trackingsysteme, kann mithilfe sogenannter \textit{Body Sensor Networks} behoben werden.
Bei diesen handelt es sich um Sensornetzwerke, welche meist aus mehreren Sensorknoten (SN) und einem Masterknoten (MN) bestehen.
Innerhalb dieses Netzwerks ist sowohl die Kommunikation der SN untereinander, als auch zwischen einzelnen SN und dem MN möglich.
Dadurch, dass die SN am Körper des Patienten positioniert werden, ist es diesem weitestgehend möglich, seinem Alltag ohne große Einschränkungen nachzugehen.
Hierdurch wird die Möglichkeit zur Langzeitanalyse verschiedener Parameter eines Patienten, außerhalb eines Labors, erheblich vergrößtert.

\subsubsection{IPANEMA Body Sensor Network}
Beim \textit{Integrated Posture and Activity NEtwork by MedIT Aachen} (IPANEMA) BSN handelt es sich um ein am MedIT entwickeltes, modulares Sensornetzwerk.


\section{Mathematische Grundlagen}
Um die angestrebte Präzision bei der Bestimmung der Positionsbeziehungen des Sensorsystems erreichen zu können, ist die Verwendung einiger mathematischer Ansätze von Nöten.
Unter anderem sind für die Filterung der durch die Sensoren aufgezeichneten Daten, der Kalman Filter, und auch eine seiner Weitereinwicklungen, der Unscented Kalman Filter, von Bedeutung.

\subsection{Kalman Filter}


\subsubsection{Unscented Kalman Filter}
